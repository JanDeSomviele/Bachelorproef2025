%%=============================================================================
%% Methodologie
%%=============================================================================

\chapter{\IfLanguageName{dutch}{Methodologie}{Methodology}}%
\label{ch:methodologie}

%% TODO: In dit hoofstuk geef je een korte toelichting over hoe je te werk bent
%% gegaan. Verdeel je onderzoek in grote fasen, en licht in elke fase toe wat
%% de doelstelling was, welke deliverables daar uit gekomen zijn, en welke
%% onderzoeksmethoden je daarbij toegepast hebt. Verantwoord waarom je
%% op deze manier te werk gegaan bent.
%% 
%% Voorbeelden van zulke fasen zijn: literatuurstudie, opstellen van een
%% requirements-analyse, opstellen long-list (bij vergelijkende studie),
%% selectie van geschikte tools (bij vergelijkende studie, "short-list"),
%% opzetten testopstelling/PoC, uitvoeren testen en verzamelen
%% van resultaten, analyse van resultaten, ...
%%
%% !!!!! LET OP !!!!!
%%
%% Het is uitdrukkelijk NIET de bedoeling dat je het grootste deel van de corpus
%% van je bachelorproef in dit hoofstuk verwerkt! Dit hoofdstuk is eerder een
%% kort overzicht van je plan van aanpak.
%%
%% Maak voor elke fase (behalve het literatuuronderzoek) een NIEUW HOOFDSTUK aan
%% en geef het een gepaste titel.
\chapter{\IfLanguageName{dutch}{Methodologie}{Methodology}}%
\label{ch:methodologie}

\section{Literatuuronderzoek en probleemdefinitie}

Het onderzoek startte met een literatuurstudie over de technische kenmerken, toepassingen en beperkingen van mobiele netwerken (4G, privaat 5G en publiek 5G) in het kader van gebouwbeheersystemen. Er werd aandacht besteed aan latentie, bandbreedte, betrouwbaarheid en beveiligingsmechanismen binnen elk netwerktype. Parallel werd ook onderzocht hoe deze eigenschappen zich verhouden tot de vereisten van toepassingen zoals verlichting en HVAC, en wat de impact kan zijn op operationele stabiliteit.

Daarnaast werden gesprekken gevoerd met IT- en facilitair personeel van HOGENT om inzicht te krijgen in de huidige architectuur, gebruikte protocollen (zoals Modbus TCP), en gevoeligheden rond beveiliging en uptime. Deze fase resulteerde in een requirementsdocument dat de basis vormde voor het opstellen van relevante testscenario’s.

\section{Ontwerp en realisatie van de testomgeving}

Aangezien de fysieke SpaceLogic AS-P controller initieel niet beschikbaar was, werd gekozen voor een flexibele testomgeving op basis van een Raspberry Pi 4 Model B. Deze fungeert als centrale testnode en voert zowel netwerkscripts als Modbus-simulaties uit. In de testopstelling wordt gebruik gemaakt van:

\begin{itemize}
    \item Een Raspberry Pi 4 met Python-scripts voor Modbus-communicatie, netwerktesten en logging;
    \item Een RUTX50-router voor mobiele connectiviteit (4G, 5G NSA/SA);
    \item Een laptop met Modbus TCP-server (simulator) voor het nabootsen van HVAC- of verlichtingsapparatuur;
    \item Mobiele netwerken: Wi-Fi, publiek 5G, privaat 5G (5Gblue) en 4G via HOGENT-infrastructuur;
    \item Testtools zoals \texttt{ping}, \texttt{speedtest-cli}, \texttt{curl}, \texttt{pymodbus}, \texttt{iPerf3}, en \texttt{psutil}.
\end{itemize}

De scripts zijn ontworpen om reproduceerbare metingen uit te voeren en loggen zowel systeemprestaties als netwerkdata in CSV-formaat.

\section{Connectiviteits- en performantiemetingen}

Voor elke netwerkomgeving (Wi-Fi, 4G, publiek 5G, privaat 5G) worden identieke metingen uitgevoerd. De scripts testen volgende aspecten:

\begin{itemize}
    \item \textbf{Latency en jitter}: via \texttt{ping}-metingen naar lokale en externe servers;
    \item \textbf{Throughput (bandbreedte)}: via \texttt{speedtest-cli} en \texttt{iPerf3};
    \item \textbf{Packet loss en stabiliteit}: over langere perioden via herhaalde communicatie;
    \item \textbf{HTTP-performance}: via \texttt{curl} om HTML-pagina’s of API-data op te halen;
    \item \textbf{Modbus-prestaties}: polling van registers op de slave-server om communicatievertragingen te detecteren;
    \item \textbf{Smart lighting responsiveness}: schakelen van een Philips Hue-lamp via een lokale bridge en logging van responstijden.
\end{itemize}

De resultaten van deze testen geven inzicht in hoe het netwerkgedrag verschilt tussen technologieën, en hoe dit zich vertaalt naar functionele impact op gebouwbeheerapplicaties.

\section{Scenario-uitwerking: worst-case en realistische belasting}

Om het effect van netwerkfluctuaties te onderzoeken, worden verschillende testscenario’s gesimuleerd:

\begin{itemize}
    \item Standaard communicatie zonder achtergrondverkeer;
    \item Zware netwerklast met gelijktijdige dataverzoeken;
    \item Simulatie van packet loss of netwerkvertraging;
    \item Verplaatsing van de testnode binnen het dekkingsgebied om signaalvariatie te introduceren.
\end{itemize}

Hierbij wordt ook een onderscheid gemaakt tussen toepassingen die sterk tijdsgevoelig zijn (bv. lichtschakelaars) versus minder kritieke processen (bv. temperatuurmetingen).

\section{Data-analyse en evaluatie}

Alle meetdata wordt automatisch opgeslagen in CSV-bestanden en met behulp van Python verwerkt tot grafieken en samenvattende tabellen. De kern van de evaluatie ligt in het vergelijken van de netwerktypes op vlak van:

\begin{itemize}
    \item Gemiddelde en maximale latency;
    \item Stabiliteit (jitter, packet loss);
    \item Reactietijd van toepassingen (bv. lamp schakeltijd, Modbus-respons);
    \item Uptime en verbindingszekerheid;
    \item Beperkingen of storingen tijdens simulaties.
\end{itemize}

De analyse wordt zowel numeriek als visueel ondersteund en vormt de basis voor de aanbevelingen in het slothoofdstuk.

\section{Onderzoeksafbakening en beperkingen}

Hoewel de testopstelling representatief is voor realistische omstandigheden, zijn enkele beperkingen van toepassing:

\begin{itemize}
    \item De gebruikte Modbus-simulatie is niet identiek aan commerciële gebouwbeheersystemen;
    \item De publieke 5G-test gebeurde via een beperkt aantal providers en locaties;
    \item Er werd geen real-time failover getest tussen netwerktypes;
    \item Beveiliging werd hoofdzakelijk theoretisch behandeld, niet op pakketniveau gemeten.
\end{itemize}

Deze beperkingen worden in de bespreking meegenomen bij de interpretatie van de resultaten en de algemene toepasbaarheid van de bevindingen voor HOGENT.




