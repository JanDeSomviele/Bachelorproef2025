%%=============================================================================
%% Methodologie
%%=============================================================================

\chapter{\IfLanguageName{dutch}{Methodologie}{Methodology}}%
\label{ch:methodologie}

%% TODO: In dit hoofstuk geef je een korte toelichting over hoe je te werk bent
%% gegaan. Verdeel je onderzoek in grote fasen, en licht in elke fase toe wat
%% de doelstelling was, welke deliverables daar uit gekomen zijn, en welke
%% onderzoeksmethoden je daarbij toegepast hebt. Verantwoord waarom je
%% op deze manier te werk gegaan bent.
%% 
%% Voorbeelden van zulke fasen zijn: literatuurstudie, opstellen van een
%% requirements-analyse, opstellen long-list (bij vergelijkende studie),
%% selectie van geschikte tools (bij vergelijkende studie, "short-list"),
%% opzetten testopstelling/PoC, uitvoeren testen en verzamelen
%% van resultaten, analyse van resultaten, ...
%%
%% !!!!! LET OP !!!!!
%%
%% Het is uitdrukkelijk NIET de bedoeling dat je het grootste deel van de corpus
%% van je bachelorproef in dit hoofstuk verwerkt! Dit hoofdstuk is eerder een
%% kort overzicht van je plan van aanpak.
%%
%% Maak voor elke fase (behalve het literatuuronderzoek) een NIEUW HOOFDSTUK aan
%% en geef het een gepaste titel.

\section{Literatuuronderzoek en probleemdefinitie}
Het doel van deze fase is om inzicht te krijgen in de huidige situatie en de technische vereisten van de systemen voor verwarming en verlichting op HOGENT-locaties. Hiervoor wordt een literatuurstudie uitgevoerd om de technische mogelijkheden en beperkingen van 4G, privaat 5G en publiek 5G te analyseren voor verwarming en verlichting. Ook worden veiligheidsrisico's bij het gebruik van mobiele netwerken onderzocht en hoe deze kunnen vermeden worden. Daarnaast worden gesprekken gevoerd met belanghebbenden bij HOGENT, zoals technisch personeel en de IT-afdeling, om een requirementsanalyse op te stellen die inzicht biedt in de huidige netwerkvereisten en mogelijke aanpassingen voor de bestaande systemen. De data voor deze fase wordt verzameld uit academische literatuur via Google Scholar, technische documentatie van HOGENT en input van interne belanghebbenden. De deliverable is een rapport waarin de technische vereisten en veiligheidsrisico’s worden gedefinieerd, en de benodigde aanpassingen om de huidige systemen compatibel te maken met mobiele netwerken.

\section{Vergelijkende studie van netwerktechnologieën}
In deze fase wordt een vergelijkende studie uitgevoerd om te beoordelen welke netwerktechnologie het meest geschikt is voor de diensten verwarming en verlichting op HOGENT. Dit omvat een evaluatie van de technische eisen die de apparaten en applicaties momenteel van het netwerk vragen. Daarnaast gebeurt een analyse van de gevolgen van een overstap naar een mobiel netwerk voor operationele continuïteit, onderhoud en beheer voor de gekozen toepassingen. Data voor deze fase wordt verzameld uit literatuur, simulaties met tools zoals Cisco Packet Tracer en experimentele data uit de 5G-omgeving van de campus Schoonmeersen. De deliverable is een rapport met een uitgebreide analyse van de geschiktheid van 4G, privaat 5G en publiek 5G, inclusief aanbevelingen voor locatiespecifieke netwerkmogelijkheden.

\section{Simulaties en proof of concept}
Deze fase richt zich op het valideren van de theoretische bevindingen door middel van simulaties en praktische tests. Netwerkprestaties worden gesimuleerd in scenario's zoals bijvoorbeeld het aansturen van slimme thermostaten en dynamische verlichting op verschillende campuslocaties. Hierbij wordt de operationele continuïteit van de diensten onderzocht en worden de vereisten voor onderhoud en beheer geëvalueerd. Daarnaast wordt een proof of concept uitgevoerd in de 5G-omgeving van de campus Schoonmeersen, waarbij IoT-apparaten voor verwarming en verlichting worden getest op compatibiliteit en prestaties in mobiele netwerken. De benodigde data omvat simulatiegegevens, resultaten van de proof of concept en specificaties van de gebruikte apparaten. De deliverable van deze fase is een analyse van de resultaten van de simulaties en de proof of concept voor de haalbaarheid en impact van de overstap naar mobiele netwerken.

\section{Data-analyse en aanbevelingen}
De laatste fase beantwoordt de overkoepelende onderzoeksvraag en de deelvragen. De verzamelde data uit de eerdere fasen wordt geanalyseerd om concrete aanbevelingen te formuleren over welke netwerktechnologie het meest geschikt is voor verwarming en verlichting op HOGENT. De deliverable van deze fase is een eindrapport met gedetailleerde aanbevelingen en implementatiestrategieën waarmee HOGENT een geïnformeerde beslissing kan nemen over de netwerkkeuzes voor haar facilitaire diensten.
