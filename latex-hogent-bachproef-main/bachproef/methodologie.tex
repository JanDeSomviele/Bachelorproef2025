%%=============================================================================
%% Methodologie
%%=============================================================================

\chapter{\IfLanguageName{dutch}{Methodologie}{Methodology}}%
\label{ch:methodologie}

%% TODO: In dit hoofstuk geef je een korte toelichting over hoe je te werk bent
%% gegaan. Verdeel je onderzoek in grote fasen, en licht in elke fase toe wat
%% de doelstelling was, welke deliverables daar uit gekomen zijn, en welke
%% onderzoeksmethoden je daarbij toegepast hebt. Verantwoord waarom je
%% op deze manier te werk gegaan bent.
%% 
%% Voorbeelden van zulke fasen zijn: literatuurstudie, opstellen van een
%% requirements-analyse, opstellen long-list (bij vergelijkende studie),
%% selectie van geschikte tools (bij vergelijkende studie, "short-list"),
%% opzetten testopstelling/PoC, uitvoeren testen en verzamelen
%% van resultaten, analyse van resultaten, ...
%%
%% !!!!! LET OP !!!!!
%%
%% Het is uitdrukkelijk NIET de bedoeling dat je het grootste deel van de corpus
%% van je bachelorproef in dit hoofstuk verwerkt! Dit hoofdstuk is eerder een
%% kort overzicht van je plan van aanpak.
%%
%% Maak voor elke fase (behalve het literatuuronderzoek) een NIEUW HOOFDSTUK aan
%% en geef het een gepaste titel.

\section{Literatuuronderzoek en probleemdefinitie}
Het doel van deze fase is om inzicht te krijgen in de huidige situatie en de technische vereisten van de systemen voor verwarming en verlichting op HOGENT-locaties. Hiervoor wordt een literatuurstudie uitgevoerd om de technische mogelijkheden en beperkingen van 4G, privaat 5G en publiek 5G te analyseren voor verwarming en verlichting. Ook worden veiligheidsrisico's bij het gebruik van mobiele netwerken onderzocht en hoe deze kunnen vermeden worden. Daarnaast worden gesprekken gevoerd met belanghebbenden bij HOGENT, zoals technisch personeel en de IT-afdeling, om een requirementsanalyse op te stellen die inzicht biedt in de huidige netwerkvereisten en mogelijke aanpassingen voor de bestaande systemen. De data voor deze fase wordt verzameld uit academische literatuur via Google Scholar, technische documentatie van HOGENT en input van interne belanghebbenden. De deliverable is een rapport waarin de technische vereisten en veiligheidsrisico’s worden gedefinieerd, en de benodigde aanpassingen om de huidige systemen compatibel te maken met mobiele netwerken.


\subsection{Basisopstelling en functionele tests}

De eerste stap bestaat uit het opzetten van een functionele testomgeving waarin de communicatie tussen HVAC/lighting modules en de netwerkcontroller gecontroleerd wordt via een stabiele Wi-Fi-verbinding. Dit gebeurt met een eenvoudige basisopstelling: een Schneider Electric SpaceLogic AS-P controller is verbonden met één of meerdere TAC Xenta 401-modules via Modbus. Deze opstelling wordt getest op een eigen laptop of pc, zonder externe netwerkinfrastructuur.

Om de connectiviteit en betrouwbaarheid van het systeem te beoordelen, worden Python-scripts ingezet die regelmatig Modbus-polling uitvoeren naar de Xenta-modules. Tegelijkertijd verzamelen deze scripts netwerkstatistieken zoals latency, packet loss en responstijd. Deze metingen worden lokaal bewaard voor latere analyse. Door de scripts eerst op Wi-Fi uit te voeren, wordt een referentiewaarde bepaald in een gecontroleerde en storingsvrije omgeving.

\subsection{Mobiele netwerken en testinfrastructuur}

Nadat de functionele werking via Wi-Fi bevestigd is, wordt de basisopstelling uitgebreid met connectiviteit via mobiele netwerken. De AS-P controller of een tussenliggende laptop wordt via een industriële 5G-router verbonden met het private 4G- of 5G-netwerk op campus Schoonmeersen. Zowel de privaat beheerde 4G-infrastructuur als de aanwezige 5G standalone-infrastructuur van het project 5Gblue worden benut.

De scripts worden opnieuw ingezet, nu met verkeer dat loopt via de mobiele netwerken. De latency, jitter en packet loss worden onder deze omstandigheden gemeten en vergeleken met de referentiewaarden van de Wi-Fi-tests. Op deze manier kan de impact van netwerkvertragingen en verbindingskwaliteit op HVAC- en verlichtingssturing objectief beoordeeld worden.

\subsection{Uitgebreide testopstelling en uitbreidbaarheid}

Naast de basisopstelling is er een optionele uitbreiding voorzien met Arduino- en radiosoftware om extra sensoren of sturingen te simuleren. Deze modules worden aangesloten via USB op een laptop en simuleren inkomende data- of commandoverkeer. Dit laat toe om de AS-P controller te testen onder realistische werklastscenario’s, zoals simultane metingen of commando's van meerdere zones in een gebouw.

Bij complexe setups of intensieve logging wordt ook een kleine server gebruikt in het netwerk. Deze server vervult meerdere rollen: centrale datalogger, visualisatiepunt, en simultane traffic generator. De server draait naast de scripts ook services voor netwerkmonitoring, zoals ping-tracing of logging van verkeer via tools als Wireshark of iPerf3.

\subsection{Analyse en evaluatie}

Alle meetdata wordt opgeslagen in CSV-formaat en verwerkt via Python voor grafische analyse. Vergelijkingen tussen 4G, 5G en Wi-Fi worden gemaakt op basis van gemiddelde responstijden, maximale vertragingen en stabiliteit van de verbinding. Tevens wordt nagegaan of netwerkvertragingen invloed hebben op de correcte werking van de HVAC- en lichtmodules. Bijvoorbeeld, worden opdrachten vertraagd uitgevoerd of niet correct verwerkt?

De resultaten van deze analyses worden samengebracht in het eindrapport, waarin ook aanbevelingen worden geformuleerd voor mogelijke implementatie van mobiele netwerken voor gebouwsturing op HOGENT. Zowel functionele als prestatiegerichte criteria worden hierbij in rekening gebracht.



