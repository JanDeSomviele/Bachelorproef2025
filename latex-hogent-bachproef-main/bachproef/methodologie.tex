%%=============================================================================
%% Methodologie
%%=============================================================================

\chapter{\IfLanguageName{dutch}{Methodologie}{Methodology}}%
\label{ch:methodologie}

%% TODO: In dit hoofstuk geef je een korte toelichting over hoe je te werk bent
%% gegaan. Verdeel je onderzoek in grote fasen, en licht in elke fase toe wat
%% de doelstelling was, welke deliverables daar uit gekomen zijn, en welke
%% onderzoeksmethoden je daarbij toegepast hebt. Verantwoord waarom je
%% op deze manier te werk gegaan bent.
%% 
%% Voorbeelden van zulke fasen zijn: literatuurstudie, opstellen van een
%% requirements-analyse, opstellen long-list (bij vergelijkende studie),
%% selectie van geschikte tools (bij vergelijkende studie, "short-list"),
%% opzetten testopstelling/PoC, uitvoeren testen en verzamelen
%% van resultaten, analyse van resultaten, ...
%%
%% !!!!! LET OP !!!!!
%%
%% Het is uitdrukkelijk NIET de bedoeling dat je het grootste deel van de corpus
%% van je bachelorproef in dit hoofstuk verwerkt! Dit hoofdstuk is eerder een
%% kort overzicht van je plan van aanpak.
%%
%% Maak voor elke fase (behalve het literatuuronderzoek) een NIEUW HOOFDSTUK aan
%% en geef het een gepaste titel.

\section{Literatuuronderzoek en probleemdefinitie}

Het onderzoek startte met een literatuurstudie over 5G netwerken, smart gebouwen/campussen en de technische kenmerken, toepassingen en beperkingen van mobiele netwerken (4G, privaat 5G en wifi) in het kader van gebouwbeheersystemen. Er werd aandacht besteed aan latentie, bandbreedte en betrouwbaarheid binnen elk netwerktype. Parallel werd ook onderzocht hoe deze eigenschappen zich verhouden tot de vereisten van toepassingen zoals verlichting en HVAC, en wat de impact kan zijn op operationele stabiliteit.
Daarnaast werden gesprekken gevoerd met personeel van de directies IT- en facilitair beheer van HOGENT om inzicht te krijgen in de huidige architectuur, gebruikte protocollen , en gevoeligheden rond beveiliging en uptime van de gebouwenbeheerinfrastructuur. Deze fase resulteerde in een literatuurstudie en onderbouwing voor de testscenario’s.


\section{Ontwerp en realisatie van de testomgeving}

De testomgeving heeft als doel het gebouwbeheersysteem te simuleren. Omdat in HOGENT een fysieke SpaceLogic AS-P controller niet beschikbaar is voor experimentele doeleinden, wordt een alternatieve en flexibele testomgeving opgezet, gebaseerd op een Raspberry Pi 4 Model B. Deze microcomputer fungeert als centrale testnode en neemt de rol van edgecontroller op zich binnen de testopstelling. 
De keuze voor dit platform is gebaseerd op de brede ondersteuning van netwerkprotocollen, de programmeerbaarheid via Python en de mogelijkheid tot headless operaties en lokale datalogging. De Raspberry Pi voert daarbij zowel netwerkmetingen als simulaties van communicatie in gebouwbeheersystemen uit.
De Pi voert netwerktests uit met behulp van tools zoals ping, node-red, curl en iPerf3. Alle resultaten worden gelogd en bewaard om later te gebruiken.
Voor mobiele netwerkconnectiviteit wordt een industriële Teltonika RUTX50-router ingezet. Deze ondersteunt zowel 4G als 5G (zowel NSA als SA), en vormt de brug tussen de testnode en het externe netwerk. De combinatie van deze router met de Raspberry Pi maakt het mogelijk om in de verschillende netwerkomstandigheden metingen uit te voeren.
Een PC doet dienst als iperf3 TCP-server en simuleert een iperf server. Deze server reageert op de verzoeken van de Raspberry Pi, zodat dit een representatieve testomgeving creëert voor communicatie in een smart building-context.
De testomgeving omvat verschillende netwerken: wifi via een lokaal access point, privaat 4G en 5G via de RUTX50 router. Deze variatie biedt de mogelijkheid om met de testopstelling netwerkprestaties te vergelijken.
Deze opstelling vormt een goed alternatief voor fysieke gebouwbeheersysteem-hardware en is aldus geschikt voor experimenteel onderzoek naar IoT-communicatie binnen slimme gebouwen.


\section{Connectiviteits- en performantiemetingen}

In elk van de netwerkomgevingen wordt een identieke testreeks uitgevoerd. Het doel is om te evalueren hoe verschillende technologieën presteren op vlak van snelheid, betrouwbaarheid en stabiliteit.
De latency en jitter worden gemeten met behulp van het ping-commando . Voor de meting van throughput wordt gebruikgemaakt van zowel speedtest-cli, dat communiceert met publieke servers, als iPerf3, waarbij een eigen server op de pc fungeert als eindpunt.
Packet loss en stabiliteit worden geëvalueerd door langdurige communicatieopdrachten te herhalen. Zo voert de Raspberry Pi HTTP-verzoeken uit met curl naar een lokale Node-RED-server en registreert daarbij responstijden en eventuele fouten. 
Een extra test meet de reactietijd van een smart verlichting-toepassing. Via een lokaal netwerk wordt een Philips Hue-lamp twee keer geschakeld door een Python-script. De tijd tussen het commando en de feitelijke uitvoering wordt gelogd. Deze test illustreert concreet hoe netwerkomstandigheden de gebruikerservaring van een IoT-toepassing kunnen beïnvloeden.


\section{Uitvoer van de testscenario’s}

TO DO: uitleggen


\section{Data-analyse en evaluatie}

TO DO: anders
Alle verzamelde data wordt automatisch verwerkt met Python. De scripts analyseren latency, jitter, packet loss, doorvoersnelheden en systeembelasting. Deze gegevens worden omgezet naar visuele representaties zoals tijdreeksen, boxplots en gemiddeldevergelijkingen, wat helpt bij het identificeren van trends of anomalieën.
De evaluatie focust op prestatieverschillen tussen de onderzochte netwerktechnologieën - Wi-Fi, 4G en privaat 5G. Daarbij worden onder meer gemiddelde en maximale latency, jitterwaarden, en de frequentie van communicatieproblemen geanalyseerd. Ook wordt expliciet nagegaan hoe snel en betrouwbaar toepassingen reageren: van het schakelen van een lamp tot het succesvol polsen van Modbus-registers.
Naast deze kwantitatieve metingen wordt ook aandacht besteed aan operationele storingen of anomalieën zoals time-outs, verbindingsverlies of onverwacht gedrag tijdens de tests. Deze kwalitatieve observaties bieden inzicht in de betrouwbaarheid van de netwerken in dagelijkse toepassingen.







