\chapter{\IfLanguageName{dutch}{scripts}{scripts}}%
\label{ch:scripts}

In dit hoofdstuk wordt dieper ingegaan op de scripts die gebruikt worden in het kader van dit onderzoek. De scripts zijn ontworpen om netwerkmetingen en Modbus-communicatie te testen tussen een Raspberry Pi en een Modbus-simulator binnen een gecontroleerde netwerkopstelling. Deze scripts dienen in eerste instantie om de betrouwbaarheid van het netwerkverkeer te testen alvorens over te gaan tot het gebruik van een fysiek gebouwbeheersysteem.

De Raspberry Pi fungeert hierbij als testplatform en voert alle scripts uit. Dankzij zijn ondersteuning voor Python en bash is de Pi een uitstekende keuze voor het automatiseren van netwerkmetingen en communicatieprotocollen.

\section{Testomgeving}

Alle scripts worden eerst getest op een lokaal WiFi-netwerk, verbonden met een standaard thuisrouter. Dit laat toe om fouten te detecteren in een vertrouwde omgeving, zonder invloed van complexe netwerkinfrastructuur zoals 5G of private netwerken.

\section{Beschrijving van de scripts}

\subsection{Netwerkmetingen (Python)}
Er wordt gebruik gemaakt van een Python-script dat netwerkstatistieken verzamelt zoals:

\begin{itemize}
    \item Ping-tijd (latency)
    \item Packet loss
    \item Jitter
    \item Bandbreedte
\end{itemize}

Modules: `ping3`, `psutil`, `speedtest-cli`

Gebruik:
\begin{minted}{bash}
    python network_metrics.py --target 8.8.8.8 --interval 5 --duration 60
\end{minted}

network\_metrics.py:
\begin{minted}{Python}
    # network_metrics.py
    import argparse
    import csv
    import time
    from ping3 import ping
    import psutil
    import speedtest
    
    def collect_metrics(target, interval, duration):
    start_time = time.time()
    results = []
    
    while (time.time() - start_time) < duration:
    latency = ping(target, unit='ms')
    timestamp = time.strftime('%Y-%m-%d %H:%M:%S')
    cpu = psutil.cpu_percent()
    mem = psutil.virtual_memory().percent
    results.append([timestamp, latency, cpu, mem])
    print(f"{timestamp} | Latency: {latency:.2f} ms | CPU: {cpu}% | RAM: {mem}%")
    time.sleep(interval)
    
    return results
    
    def write_csv(results, filename="network_results.csv"):
    with open(filename, mode='w', newline='') as file:
    writer = csv.writer(file)
    writer.writerow(['Timestamp', 'Latency (ms)', 'CPU (%)', 'RAM (%)'])
    
\end{minted}

Parameters:
\begin{itemize}
    \item `--target`: het IP-adres dat gepingd moet worden (bv. een publieke DNS-server)
    \item `--interval`: tijd in seconden tussen metingen
    \item `--duration`: totale duur van de test in seconden
\end{itemize}

De resultaten worden gelogd in een CSV-bestand voor latere analyse.

\subsection{Modbus TCP Test (Python)}

Er wordt gebruik gemaakt van de `pymodbus` bibliotheek om een Modbus TCP-client te simuleren die connecteert naar een lokale of externe Modbus TCP-server. Dit simuleert de communicatie zoals die zou verlopen tussen een HVAC-controller en sensoren of actuatoren.

Gebruik:
\begin{minted}{bash}
    python modbus_poll.py --host 127.0.0.1 --port 5020 --unit 1 --count 10
\end{minted}

modbus\_poll.py:
\begin{minted}{Python}
    # modbus_poll.py
    import argparse
    from pymodbus.client import ModbusTcpClient
    
    def poll_modbus(host, port, unit, count):
    client = ModbusTcpClient(host, port=port)
    if not client.connect():
    print("Fout: Kon geen verbinding maken met Modbus-server.")
    return
    
    rr = client.read_holding_registers(0, count, unit=unit)
    if rr.isError():
    print(f"Fout bij het lezen van registers: {rr}")
    else:
    print(f"Gelezen waarden: {rr.registers}")
    
    client.close()
    
    def main():
    parser = argparse.ArgumentParser()
    parser.add_argument('--host', default='127.0.0.1', help='Modbus TCP server IP')
    parser.add_argument('--port', type=int, default=5020, help='TCP poort')
    parser.add_argument('--unit', type=int, default=1, help='Modbus unit ID')
    parser.add_argument('--count', type=int, default=10, help='Aantal registers om te lezen')
    args = parser.parse_args()
    
    poll_modbus(args.host, args.port, args.unit, args.count)
    
    if __name__ == "__main__":
    main()
    
\end{minted}

Parameters:
\begin{itemize}
    \item `--host`: IP-adres van de Modbus server (kan lokaal of op een ander apparaat zijn)
    \item `--port`: TCP-poort van de server (standaard: 502 of 5020)
    \item `--unit`: Modbus unit ID
    \item `--count`: aantal registers om uit te lezen
\end{itemize}

De Modbus-server wordt lokaal op de Raspberry Pi gestart via een simulator of op een aparte computer binnen hetzelfde netwerk.

\subsection{Logging en automatisatie (Bash)}

Met behulp van bash-scripts worden de Python-scripts automatisch gestart, logbestanden georganiseerd en periodieke testen ingepland. Dit biedt de mogelijkheid om testen te herhalen over langere tijdsperiodes.

run\_tests.sh:

\begin{minted}{Bash}
    #!/bin/bash
    mkdir -p logs
    DATE=$(date +"%Y%m%d_%H%M%S")
    
    echo "Start netwerkmeting..."
    python3 network_metrics.py --target 8.8.8.8 --interval 5 --duration 60 > logs/network_$DATE.csv
    
    echo "Start Modbus polling..."
    python3 modbus_poll.py --host 127.0.0.1 --port 5020 --unit 1 --count 10 > logs/modbus_$DATE.txt
    
    echo "Tests afgerond. Logs opgeslagen in ./logs/"
    
\end{minted}



%\section{Toekomstige uitbreiding}
%In een latere fase kunnen deze scripts ook worden aangepast om communicatie te testen met een echte HVAC-controller (zoals de TAC Xenta of een SpaceLogic SmartX AS-P), zodra deze beschikbaar is. Momenteel vormt de Raspberry Pi de kern van de testopstelling, met gesimuleerde apparaten om een representatief communicatieprofiel op te bouwen.

