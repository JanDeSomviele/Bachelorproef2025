%%=============================================================================
%% Voorwoord
%%=============================================================================

\chapter*{\IfLanguageName{dutch}{Woord vooraf}{Preface}}%
\label{ch:voorwoord}

%% TODO:
%% Het voorwoord is het enige deel van de bachelorproef waar je vanuit je
%% eigen standpunt (``ik-vorm'') mag schrijven. Je kan hier bv. motiveren
%% waarom jij het onderwerp wil bespreken.
%% Vergeet ook niet te bedanken wie je geholpen/gesteund/... heeft

%TO DO: begin anders doen
%Toen ik aan mijn bachelorproef begon, wist ik dat ik een onderwerp wou kiezen dat niet alleen technisch uitdagend was, maar ook sterk verbonden met actuele technologische evoluties. De snelle opmars van private 5G-netwerken en hun potentieel in industriële toepassingen wekten al langer mijn interesse. Het idee om deze technologie toe te passen op bestaande HVAC- en verlichtingssystemen binnen HOGENT bood mij een unieke kans om praktijk en theorie samen te brengen. Deze bachelorproef gaf me de mogelijkheid om mijn vaardigheden op vlak van netwerken, automatisering en dataverzameling verder te verdiepen in een realistische context.

Dit traject had ik echter niet alleen kunnen afleggen. Ik wil hierbij enkele mensen oprecht bedanken voor hun waardevolle steun en begeleiding. Van HOGENT wil ik in het bijzonder Tom Venneman, Piet Standaert, Jeroen Goethals, Lena De Mol en Giselle Vercauteren bedanken voor hun technische input, kritische feedback en bereidheid om hun tijd en kennis te delen. Van CITYMESH wens ik Tom Collins te bedanken voor het aanreiken van inzichten rond private 5G-infrastructuur en de mogelijkheden ervan binnen facilitaire toepassingen.

Ten slotte wil ik mijn familie bedanken voor hun onvoorwaardelijke steun en aanmoediging tijdens dit project. Hun geloof in mijn kunnen en hun geduld tijdens drukke en stressvolle periodes betekenden veel voor mij.

%TO DO: einde
%Deze bachelorproef is het resultaat van vele uren werk, onderzoek en samenwerking, en ik hoop dat de inhoud ervan een meerwaarde kan betekenen voor de verdere ontwikkeling van moderne netwerkinfrastructuren binnen onderwijsinstellingen zoals HOGENT.