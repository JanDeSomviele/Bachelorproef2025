%%=============================================================================
%% Voorwoord
%%=============================================================================

\chapter*{\IfLanguageName{dutch}{Woord vooraf}{Preface}}%
\label{ch:voorwoord}

%% TODO:
%% Het voorwoord is het enige deel van de bachelorproef waar je vanuit je
%% eigen standpunt (``ik-vorm'') mag schrijven. Je kan hier bv. motiveren
%% waarom jij het onderwerp wil bespreken.
%% Vergeet ook niet te bedanken wie je geholpen/gesteund/... heeft
Dit traject voor het maken van mijn bachelorproef had ik niet alleen kunnen afleggen. Ik wil hierbij enkele mensen oprecht bedanken voor hun waardevolle steun en begeleiding.
\newline
\newline
Van HOGENT wil ik graag Tom Venneman, Piet Standaert, Jeroen Goethals, Lena De Mol en Giselle Vercauteren bedanken voor hun technische input, kritische feedback en bereidheid om hun tijd en kennis te delen. Van Citymesh wens ik in het bijzonder Tom Collins te bedanken voor het aanreiken van inzichten rond private 5G-infrastructuur en de mogelijkheden ervan binnen facilitaire toepassingen. Ik wil ook graag Dries De Groote van Jansen Smart Solutions bedanken om mee na te denken over de opzet van de technische component.
\newline
\newline
Ten slotte wil ik mijn familie bedanken voor hun onvoorwaardelijke steun en aanmoediging tijdens dit project. Hun blijvend geloof in mijn kunnen en hun geduld tijdens alle drukke en stressvolle periodes betekenden veel voor mij


%TO DO: einde
%Deze bachelorproef is het resultaat van vele uren werk, onderzoek en samenwerking, en ik hoop dat de inhoud ervan een meerwaarde kan betekenen voor de verdere ontwikkeling van moderne netwerkinfrastructuren binnen onderwijsinstellingen zoals HOGENT.