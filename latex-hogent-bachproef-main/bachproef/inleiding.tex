%%=============================================================================
%% Inleiding
%%=============================================================================

\chapter{\IfLanguageName{dutch}{Inleiding}{Introduction}}%
\label{ch:inleiding}

Deze bachelorproef richt zich op het onderzoeken van de impact van de overstap van een bekabeld netwerk naar mobiele netwerken, zoals 4G, privaat 5G, en publiek 5G, voor de facilitaire diensten verwarming en verlichting van HOGENT. Doordat apparatuur steeds meer gebruik maakt van internettoegang en cloudmanagement, wordt de vraag gesteld of een mobiel netwerk een een betrouwbare en veilige oplossing kan bieden. Momenteel worden verlichting en verwarming op de campus Schoonmeersen geautomatiseerd aangestuurd via tijdschema's, sensoren en regelmodules. De overstap naar een mobiel netwerk brengt zowel voordelen als uitdagingen mee: een mobiel netwerk belooft snellere verbindingen en flexibiliteit, maar zorgt ook voor de nood aan afwegingen met betrekking tot bandbreedte, netwerkbelasting en veiligheid. De doelgroep van dit onderzoek is het faciliteir beheer van HOGENT. De centrale onderzoeksvraag van deze bachelorproef is: \textit{Welk netwerk (4G, privaat 5G of publiek 5G) biedt de beste balans tussen beveiliging, prestaties en netwerkbelasting voor de facilitaire diensten verwarming en verlichting van HOGENT?}. Naast deze onderzoeksvraag worden volgende vragen ook beantwoord: 
\begin{itemize}
    \item Welke technische vereisten hebben de systemen voor verwarming en verlichting op HOGENT-locaties?
    \item Wat zijn de veiligheidsrisico’s bij het gebruik van mobiele netwerken voor verwarming en verlichting, en hoe kunnen deze worden beperkt?
    \item Welke aanpassingen zijn nodig om de bestaande bekabelde systemen voor verwarming en verlichting compatibel te maken met mobiele netwerken?
    \item Wat zijn de implicaties van de keuze voor een mobiel netwerk op de operationele continuïteit van de diensten verwarming en verlichting?
    \item Wat zijn de implicaties van een overstap naar een mobiel netwerk voor onderhoud en beheer van de verwarmings- en verlichtingssystemen?
    \item Is het zinvol om volledig over te stappen naar een privaat 5G netwerk, en hoe kan dit het best worden gerealiseerd?
\end{itemize}
%
%De inleiding moet de lezer net genoeg informatie verschaffen om het onderwerp te begrijpen en in te zien waarom de onderzoeksvraag de moeite waard is om te onderzoeken. In de inleiding ga je literatuurverwijzingen beperken, zodat de tekst vlot leesbaar blijft. Je kan de inleiding verder onderverdelen in secties als dit de tekst verduidelijkt. Zaken die aan bod kunnen komen in de inleiding~\autocite{Pollefliet2011}:
%
%\begin{itemize}
%  \item context, achtergrond
%  \item afbakenen van het onderwerp
%  \item verantwoording van het onderwerp, methodologie
%  \item probleemstelling
%  \item onderzoeksdoelstelling
%  \item onderzoeksvraag
%  \item \ldots
%\end{itemize}
%
%\section{\IfLanguageName{dutch}{Probleemstelling}{Problem Statement}}%
%\label{sec:probleemstelling}
%
%Uit je probleemstelling moet duidelijk zijn dat je onderzoek een meerwaarde heeft voor een concrete doelgroep. De doelgroep moet goed gedefinieerd en afgelijnd zijn. Doelgroepen als ``bedrijven,'' ``KMO's'', systeembeheerders, enz.~zijn nog te vaag. Als je een lijstje kan maken van de personen/organisaties die een meerwaarde zullen vinden in deze bachelorproef (dit is eigenlijk je steekproefkader), dan is dat een indicatie dat de doelgroep goed gedefinieerd is. Dit kan een enkel bedrijf zijn of zelfs één persoon (je co-promotor/opdrachtgever).
%
%\section{\IfLanguageName{dutch}{Onderzoeksvraag}{Research question}}%
%\label{sec:onderzoeksvraag}
%
%Wees zo concreet mogelijk bij het formuleren van je onderzoeksvraag. Een onderzoeksvraag is trouwens iets waar nog niemand op dit moment een antwoord heeft (voor zover je kan nagaan). Het opzoeken van bestaande informatie (bv. ``welke tools bestaan er voor deze toepassing?'') is dus geen onderzoeksvraag. Je kan de onderzoeksvraag verder specifiëren in deelvragen. Bv.~als je onderzoek gaat over performantiemetingen, dan 
%
%\section{\IfLanguageName{dutch}{Onderzoeksdoelstelling}{Research objective}}%
%\label{sec:onderzoeksdoelstelling}
%
%Wat is het beoogde resultaat van je bachelorproef? Wat zijn de criteria voor succes? Beschrijf die zo concreet mogelijk. Gaat het bv.\ om een proof-of-concept, een prototype, een verslag met aanbevelingen, een vergelijkende studie, enz.

\section{\IfLanguageName{dutch}{Opzet van deze bachelorproef}{Structure of this bachelor thesis}}%
\label{sec:opzet-bachelorproef}

% Het is gebruikelijk aan het einde van de inleiding een overzicht te
% geven van de opbouw van de rest van de tekst. Deze sectie bevat al een aanzet
% die je kan aanvullen/aanpassen in functie van je eigen tekst.

De rest van deze bachelorproef is als volgt opgebouwd:

In Hoofdstuk~\ref{ch:stand-van-zaken} wordt een overzicht gegeven van de stand van zaken binnen het onderzoeksdomein, op basis van een literatuurstudie.

In Hoofdstuk~\ref{ch:methodologie} wordt de methodologie toegelicht en worden de gebruikte onderzoekstechnieken besproken om een antwoord te kunnen formuleren op de onderzoeksvragen.

% TODO: Vul hier aan voor je eigen hoofstukken, één of twee zinnen per hoofdstuk

In Hoofdstuk~\ref{ch:conclusie}, tenslotte, wordt de conclusie gegeven en een antwoord geformuleerd op de onderzoeksvragen. Daarbij wordt ook een aanzet gegeven voor toekomstig onderzoek binnen dit domein.