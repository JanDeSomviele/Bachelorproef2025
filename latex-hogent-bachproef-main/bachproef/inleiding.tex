%%=============================================================================
%% Inleiding
%%=============================================================================

\chapter{\IfLanguageName{dutch}{Inleiding}{Introduction}}%
\label{ch:inleiding}

Deze bachelorproef onderzoekt de haalbaarheid en meerwaarde van mobiele netwerken, in het bijzonder 4G, publiek 5G en privaat 5G, als alternatief voor bekabelde netwerkinfrastructuur bij de aansturing van facilitaire systemen zoals verlichting en HVAC (Heating, Ventilation and Air Conditioning) binnen HOGENT. Door de toenemende digitalisering en de opkomst van cloudgebaseerd beheer van gebouwfuncties, ontstaat de nood om de traditionele netwerkinfrastructuren te herzien. Mobiele netwerken beloven immers een grotere flexibiliteit, hogere snelheid en snellere uitrolmogelijkheden — maar tegelijkertijd brengen ze ook uitdagingen met zich mee op vlak van betrouwbaarheid, latency, netwerkbelasting en beveiliging.

Momenteel worden de verlichting en HVAC-systemen op de campus Schoonmeersen automatisch aangestuurd via een combinatie van tijdschema’s, aanwezigheidsdetectie en lokale besturingsmodules die communiceren over een bekabeld IP-netwerk. Binnen deze context stelt zich de vraag in welke mate mobiele netwerken, en met name een privaat 5G-netwerk, een waardig alternatief vormen. De beoogde eindgebruiker van dit onderzoek is het facilitair beheer van HOGENT, dat in het kader van toekomstige infrastructuurprojecten een geïnformeerde keuze moet kunnen maken tussen bestaande en nieuwe netwerktechnologieën.

Om deze thematiek grondig te analyseren, wordt een reeks netwerkmetingen en simulaties opgezet die zowel technische netwerkparameters als praktische gebouwbeheertoepassingen omvatten. Door middel van een testopstelling met een Raspberry Pi, een RUTX50-router en diverse scripts worden meetdata verzameld over latency, jitter, throughput, packet loss, HTTP-responsiviteit, en de performantie van smart verlichting via verschillende netwerken. Deze metingen worden telkens herhaald op 4G, publiek 5G en privaat 5G, zodat een vergelijkende analyse mogelijk is.

De centrale onderzoeksvraag van deze bachelorproef luidt als volgt:

\begin{quote}
    \textit{Wat is het verschil tussen 4G, publiek 5G en privaat 5G in verband met prestaties, betrouwbaarheid en beveiliging van toepassingen zoals verlichting en HVAC binnen een gebouwbeheersysteem?}
\end{quote}

Om deze hoofdvraag te kunnen beantwoorden, wordt ook stilgestaan bij een aantal onderliggende deelvragen:

\begin{itemize}
    \item Welke technische vereisten stellen HVAC- en verlichtingssystemen op HOGENT-locaties aan de netwerkconnectiviteit?
    \item Welke verschillen zijn er in latency, jitter, packet loss en throughput tussen 4G, publiek 5G en privaat 5G?
    \item Hoe betrouwbaar zijn netwerkinteracties (zoals HTTP-verzoeken en Modbus-communicatie) bij mobiele netwerken in een gebouwbeheerscenario?
    \item Wat zijn de implicaties van de overstap naar mobiele netwerken op de operationele continuïteit van HVAC- en verlichtingssystemen?
    \item Wat zijn de beveiligingsrisico’s van mobiele netwerken in deze context, en welke maatregelen zijn nodig om die te beperken?
    \item Welke praktische aanpassingen zijn vereist om bestaande bekabelde systemen compatibel te maken met mobiele infrastructuur?
    \item Onder welke voorwaarden is het zinvol om volledig over te stappen naar een privaat 5G-netwerk, en hoe kan een dergelijk netwerk best uitgerold worden?
\end{itemize}

Door het combineren van metingen op netwerkniveau met simulaties van concrete toepassingen (zoals smart lighting via Philips Hue en HTTP-gebaseerde dataverzoeken), biedt deze bachelorproef een geïntegreerde analyse van de geschiktheid van mobiele netwerken in het domein van gebouwbeheer.

%
%De inleiding moet de lezer net genoeg informatie verschaffen om het onderwerp te begrijpen en in te zien waarom de onderzoeksvraag de moeite waard is om te onderzoeken. In de inleiding ga je literatuurverwijzingen beperken, zodat de tekst vlot leesbaar blijft. Je kan de inleiding verder onderverdelen in secties als dit de tekst verduidelijkt. Zaken die aan bod kunnen komen in de inleiding~\autocite{Pollefliet2011}:
%
%\begin{itemize}
%  \item context, achtergrond
%  \item afbakenen van het onderwerp
%  \item verantwoording van het onderwerp, methodologie
%  \item probleemstelling
%  \item onderzoeksdoelstelling
%  \item onderzoeksvraag
%  \item \ldots
%\end{itemize}
%
%\section{\IfLanguageName{dutch}{Probleemstelling}{Problem Statement}}%
%\label{sec:probleemstelling}
%
%Uit je probleemstelling moet duidelijk zijn dat je onderzoek een meerwaarde heeft voor een concrete doelgroep. De doelgroep moet goed gedefinieerd en afgelijnd zijn. Doelgroepen als ``bedrijven,'' ``KMO's'', systeembeheerders, enz.~zijn nog te vaag. Als je een lijstje kan maken van de personen/organisaties die een meerwaarde zullen vinden in deze bachelorproef (dit is eigenlijk je steekproefkader), dan is dat een indicatie dat de doelgroep goed gedefinieerd is. Dit kan een enkel bedrijf zijn of zelfs één persoon (je co-promotor/opdrachtgever).
%
%\section{\IfLanguageName{dutch}{Onderzoeksvraag}{Research question}}%
%\label{sec:onderzoeksvraag}
%
%Wees zo concreet mogelijk bij het formuleren van je onderzoeksvraag. Een onderzoeksvraag is trouwens iets waar nog niemand op dit moment een antwoord heeft (voor zover je kan nagaan). Het opzoeken van bestaande informatie (bv. ``welke tools bestaan er voor deze toepassing?'') is dus geen onderzoeksvraag. Je kan de onderzoeksvraag verder specifiëren in deelvragen. Bv.~als je onderzoek gaat over performantiemetingen, dan 
%
%\section{\IfLanguageName{dutch}{Onderzoeksdoelstelling}{Research objective}}%
%\label{sec:onderzoeksdoelstelling}
%
%Wat is het beoogde resultaat van je bachelorproef? Wat zijn de criteria voor succes? Beschrijf die zo concreet mogelijk. Gaat het bv.\ om een proof-of-concept, een prototype, een verslag met aanbevelingen, een vergelijkende studie, enz.

\section{\IfLanguageName{dutch}{Opzet van deze bachelorproef}{Structure of this bachelor thesis}}%
\label{sec:opzet-bachelorproef}

% Het is gebruikelijk aan het einde van de inleiding een overzicht te
% geven van de opbouw van de rest van de tekst. Deze sectie bevat al een aanzet
% die je kan aanvullen/aanpassen in functie van je eigen tekst.

De rest van deze bachelorproef is als volgt opgebouwd:

In Hoofdstuk~\ref{ch:stand-van-zaken} wordt een overzicht gegeven van de stand van zaken binnen het onderzoeksdomein, op basis van een literatuurstudie.

In Hoofdstuk~\ref{ch:methodologie} wordt de methodologie toegelicht en worden de gebruikte onderzoekstechnieken besproken om een antwoord te kunnen formuleren op de onderzoeksvragen.

% TODO: Vul hier aan voor je eigen hoofstukken, één of twee zinnen per hoofdstuk

In Hoofdstuk~\ref{ch:scripts} wordt de gebruikte scripts uitgelegd.
In Hoofdstuk~\ref{ch:basisopstelling} wordt de eerste opstelling voor de testen uitgelegd.
In Hoofdstuk~\ref{ch:uitgebreide-opstelling} wordt de uitgebreide opstelling uitgelegd.
In Hoofdstuk~\ref{ch:resultaten} worden de resultaten van de testen uitgelegd.

In Hoofdstuk~\ref{ch:conclusie}, tenslotte, wordt de conclusie gegeven en een antwoord geformuleerd op de onderzoeksvragen. Daarbij wordt ook een aanzet gegeven voor toekomstig onderzoek binnen dit domein.