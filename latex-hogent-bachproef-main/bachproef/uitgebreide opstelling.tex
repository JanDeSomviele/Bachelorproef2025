\chapter{\IfLanguageName{dutch}{uitgebreide opstelling}{}}%
\label{ch:uitgebreide-opstelling}

Voor het uitvoeren van netwerkgerelateerde prestatietests op HVAC- en verlichtingssystemen in gebouw C, werd een testopstelling ontwikkeld waarin een private 4G- en 5G-netwerk vergeleken worden. Om deze tests op een gecontroleerde en schaalbare manier te kunnen uitvoeren, werd een kleine server opgenomen in het testnetwerk.

\section*{Doel van de opstelling}

Het hoofddoel van deze testopstelling is het evalueren van de performantie van communicatieprotocollen, zoals Modbus, die gebruikt worden voor het aansturen van TAC Xenta modules via een SpaceLogic AS-P controller. Er wordt onderzocht hoe deze systemen reageren onder verschillende netwerkcondities (4G vs 5G). De integratie van een lokale server maakt het mogelijk om testdata te genereren, netwerkverkeer te loggen en simulaties of scriptgestuurde polls uit te voeren.

\section*{Benodigde hardware}

\begin{itemize}
    \item \textbf{SpaceLogic AS-P controller} – fungeert als gateway tussen IP-netwerk en veldapparatuur (Modbus/RS485).
    \item \textbf{TAC Xenta modules (bv. Xenta 401)} – HVAC-controllers aangesloten via Modbus.
    \item \textbf{5G Router (bv. Teltonika RUTX50 of vergelijkbaar)} – verbindt de AS-P controller met het 5G netwerk.
    \item \textbf{4G Router} – voor testen over een privaat LTE-netwerk.
    \item \textbf{Mini Server (bv. Intel NUC of Raspberry Pi 5)} – draait scripts, verzamelt data en fungeert als testbedcontroller.
    \item \textbf{Laptop/PC} – voor initiële configuratie, monitoring en data-analyse.
    \item \textbf{Netwerkcomponenten} – Ethernetkabels, voeding, eventueel een switch.
\end{itemize}

\section*{Software en tools}

\begin{itemize}
    \item \textbf{Python/Bash scripts} – voor periodieke polling van Modbus registers en logging van reactietijden.
    \item \textbf{Grafana + InfluxDB} – voor visualisatie en opslag van netwerk- en systeemdata.
    \item \textbf{Tcpdump/iPerf} – netwerkdiagnosetools voor latency, jitter en packet loss.
    \item \textbf{Modbus client} – via Python (pymodbus) of modpoll CLI-tool.
\end{itemize}

\section*{Netwerkstructuur}

De server bevindt zich in hetzelfde subnet als de AS-P controller en is bereikbaar via zowel 4G als 5G afhankelijk van het testscenario. De Xenta modules zijn via Modbus (RTU of TCP) verbonden met de AS-P controller, die als interface naar het IP-netwerk fungeert. Alle communicatie tussen de controller en de server verloopt via de mobiele netwerken, waarbij testsoftware op de server de communicatie meet, logt en visualiseert.

\section*{Testprocedure}

\begin{enumerate}
    \item Configuratie van de AS-P: verbinding met Xenta modules opzetten, Modbus-mapping controleren.
    \item Netwerkconnectie testen: Router configureren voor respectievelijk 4G en 5G.
    \item Metingen uitvoeren:
    \begin{itemize}
        \item Latency: gemiddelde responstijd van Modbus polls.
        \item Jitter: variatie in responstijden.
        \item Packet loss: percentage verloren pakketten.
        \item Betrouwbaarheid: frequentie van time-outs of verbindingsonderbrekingen.
    \end{itemize}
    \item Data logging: met scripts wordt continu data opgeslagen op de server.
    \item Analyse \& vergelijking: met Grafana worden verschillen tussen 4G en 5G inzichtelijk gemaakt.
\end{enumerate}

\section*{Rol van de server}

De server speelt een centrale rol als controle- en dataverzamelpunt in de testopstelling. Het voordeel hiervan is dat alle meetdata lokaal opgeslagen kunnen worden, onafhankelijk van externe cloudverbindingen. Dit maakt reproduceerbare en betrouwbare metingen mogelijk en verlaagt de complexiteit van troubleshooting.


