%%=============================================================================
%% Samenvatting
%%=============================================================================

% TODO: De "abstract" of samenvatting is een kernachtige (~ 1 blz. voor een
% thesis) synthese van het document.
%
% Een goede abstract biedt een kernachtig antwoord op volgende vragen:
%
% 1. Waarover gaat de bachelorproef?
% 2. Waarom heb je er over geschreven?
% 3. Hoe heb je het onderzoek uitgevoerd?
% 4. Wat waren de resultaten? Wat blijkt uit je onderzoek?
% 5. Wat betekenen je resultaten? Wat is de relevantie voor het werkveld?
%
% Daarom bestaat een abstract uit volgende componenten:
%
% - inleiding + kaderen thema
% - probleemstelling
% - (centrale) onderzoeksvraag
% - onderzoeksdoelstelling
% - methodologie
% - resultaten (beperk tot de belangrijkste, relevant voor de onderzoeksvraag)
% - conclusies, aanbevelingen, beperkingen
%
% LET OP! Een samenvatting is GEEN voorwoord!

%%---------- Nederlandse samenvatting -----------------------------------------
%
% TODO: Als je je bachelorproef in het Engels schrijft, moet je eerst een
% Nederlandse samenvatting invoegen. Haal daarvoor onderstaande code uit
% commentaar.
% Wie zijn bachelorproef in het Nederlands schrijft, kan dit negeren, de inhoud
% wordt niet in het document ingevoegd.

\IfLanguageName{english}{%
\selectlanguage{dutch}
\chapter*{Samenvatting}
\lipsum[1-4]
\selectlanguage{english}
}{}

%%---------- Samenvatting -----------------------------------------------------
% De samenvatting in de hoofdtaal van het document

\chapter*{\IfLanguageName{dutch}{Samenvatting}{Abstract}}

Deze bachelorproef onderzoekt de overstap van een bekabeld netwerk naar mobiele netwerken zoals 4G, privaat 5G en publiek 5G voor de facilitair diensten verwarming en verlichting van HOGENT. Doordat facilitaire apparatuur steeds meer afhankelijk is van internettoegang en cloudmanagement is het nodig om een netwerk te kiezen die een goede balans biedt tussen beveiliging, prestaties en netwerkbelasting. De centrale onderzoeksvraag van deze bachelorproef is: \textit{Welk netwerk (4G, privaat 5G of publiek 5G) biedt de beste balans tussen beveiliging, prestaties en netwerkbelasting voor de facilitaire diensten verwarming en verlichting van HOGENT?} De voorgestelde methodologie omvat een literatuurstudie, een vergelijkende analyse van netwerkmogelijkheden, simulaties en een proof-of-concept uitgevoerd op de 5G-omgeving van de campus Schoonmeersen. Het verwachte resultaat van het onderzoek is dat een privaat 5G netwerk de beste balans biedt en als deze niet beschikbaar is een combinatie van 4G en 5G een passend alternatief is. Het onderzoek resulteert in een rapport met concrete aanbevelingen voor HOGENT, waarmee de netwerkkeuzes voor verwarming en verlichting verbeterd kunnen worden. De meerwaarde van deze studie ligt in het leveren van inzichten die bijdragen aan efficiëntere, veiligere en toekomstbestendige netwerkoplossingen binnen HOGENT.
