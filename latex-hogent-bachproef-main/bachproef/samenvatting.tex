%%=============================================================================
%% Samenvatting
%%=============================================================================

% TODO: De "abstract" of samenvatting is een kernachtige (~ 1 blz. voor een
% thesis) synthese van het document.
%
% Een goede abstract biedt een kernachtig antwoord op volgende vragen:
%
% 1. Waarover gaat de bachelorproef?
% 2. Waarom heb je er over geschreven?
% 3. Hoe heb je het onderzoek uitgevoerd?
% 4. Wat waren de resultaten? Wat blijkt uit je onderzoek?
% 5. Wat betekenen je resultaten? Wat is de relevantie voor het werkveld?
%
% Daarom bestaat een abstract uit volgende componenten:
%
% - inleiding + kaderen thema
% - probleemstelling
% - (centrale) onderzoeksvraag
% - onderzoeksdoelstelling
% - methodologie
% - resultaten (beperk tot de belangrijkste, relevant voor de onderzoeksvraag)
% - conclusies, aanbevelingen, beperkingen
%
% LET OP! Een samenvatting is GEEN voorwoord!

%%---------- Nederlandse samenvatting -----------------------------------------
%
% TODO: Als je je bachelorproef in het Engels schrijft, moet je eerst een
% Nederlandse samenvatting invoegen. Haal daarvoor onderstaande code uit
% commentaar.
% Wie zijn bachelorproef in het Nederlands schrijft, kan dit negeren, de inhoud
% wordt niet in het document ingevoegd.

\IfLanguageName{english}{%
\selectlanguage{dutch}
\chapter*{Samenvatting}
\lipsum[1-4]
\selectlanguage{english}
}{}

%%---------- Samenvatting -----------------------------------------------------
% De samenvatting in de hoofdtaal van het document

\chapter*{\IfLanguageName{dutch}{Samenvatting}{Abstract}}

Deze bachelorproef onderzoekt de verschillen tussen 4G en privaat 5G voor de facilitair diensten verwarming en verlichting van HOGENT. Doordat facilitaire apparatuur steeds meer afhankelijk is van internettoegang en cloudmanagement is het nodig om een netwerk te kiezen die een goede balans biedt tussen beveiliging, prestaties en netwerkbelasting. De centrale onderzoeksvraag van deze bachelorproef is: \textit{Wat is het verschil tussen 4G en privaat 5G in verband met prestaties, betrouwbaarheid en beveiliging van toepassingen zoals verlichting en HVAC binnen een gebouwbeheersysteem?} De voorgestelde methodologie omvat een literatuurstudie, een aantal testen en een vergelijkende analyse van netwerkmogelijkheden uitgevoerd op de 5G-omgeving van de campus Schoonmeersen van HOGENT. Het verwachte resultaat van het onderzoek is dat een privaat 5G netwerk de beste balans biedt en als deze niet beschikbaar is een combinatie van 4G en 5G een passend alternatief is. Het onderzoek levert waardevolle inzichten en concrete aanbevelingen op die HOGENT ondersteunen bij het optimaliseren van het 5G-netwerk binnen de campus. De resultaten bieden een stevige basis voor toekomstige onderzoeken en dragen bij aan een beter begrip van de technische mogelijkheden en uitdagingen. Hierdoor kan HOGENT weloverwogen keuzes maken voor de verdere uitrol en het beheer van een efficiënt, veilig en toekomstbestendig 5G-netwerk.
