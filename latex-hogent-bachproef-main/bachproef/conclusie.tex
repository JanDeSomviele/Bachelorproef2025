%%=============================================================================
%% Conclusie
%%=============================================================================

\chapter{Conclusie}%
\label{ch:conclusie}

% TODO: Trek een duidelijke conclusie, in de vorm van een antwoord op de
% onderzoeksvra(a)g(en). Wat was jouw bijdrage aan het onderzoeksdomein en
% hoe biedt dit meerwaarde aan het vakgebied/doelgroep? 
% Reflecteer kritisch over het resultaat. In Engelse teksten wordt deze sectie
% ``Discussion'' genoemd. Had je deze uitkomst verwacht? Zijn er zaken die nog
% niet duidelijk zijn?
% Heeft het onderzoek geleid tot nieuwe vragen die uitnodigen tot verder 
%onderzoek?


Het doel van deze bachelorproef is om te onderzoeken welk type netwerk (4G, publiek 5G of privaat 5G) het meest geschikt is voor het aansturen van HVAC- en verlichtingssystemen op de HOGENT-campus. Daarbij wordt gekeken naar netwerkprestaties, beveiliging en betrouwbaarheid.

Uit de uitgevoerde testen blijkt dat privaat 5G de beste balans biedt tussen stabiliteit, lage latency en voorspelbaar gedrag. In vergelijking met publiek 5G en 4G is het minder gevoelig voor externe factoren zoals netwerkcongestie of handovers tussen zendmasten. Ook de mogelijkheid tot volledige controle over het netwerk vormt een belangrijk voordeel in het kader van beveiliging en gegarandeerde prestaties.

Publiek 5G toont degelijke prestaties, maar blijft afhankelijk van locatie en netwerkbelasting. De resultaten zijn minder voorspelbaar, wat een risico inhoudt voor real-time toepassingen zoals verlichting. 4G blijkt bruikbaar voor minder kritieke toepassingen, maar voldoet niet aan de vereisten voor lage latency en betrouwbare beschikbaarheid.

\section{Aanbevelingen}

Op basis van de analyse van de meetresultaten worden de volgende aanbevelingen geformuleerd:

\begin{itemize}
    \item \textbf{Voorzie een gefaseerde migratie} naar privaat 5G voor kritieke gebouwsturingstoepassingen zoals verlichting en HVAC;
    \item \textbf{Beperk de risico’s} door een fallback-optie te behouden via Wi-Fi of een bekabeld netwerk;
    \item \textbf{Implementeer continue monitoring} van netwerkprestaties en applicatiegedrag via scripts en logbestanden;
    \item \textbf{Voer periodieke beveiligingsaudits uit} voor alle draadloze segmenten van het netwerk, vooral bij gebruik van publiek 5G;
    \item \textbf{Test toepassingen in realistische scenario’s}, inclusief interferentie, mobiliteit en netwerklast.
\end{itemize}

Deze studie toont aan dat mobiele netwerken technisch geschikt zijn voor gebouwsturing, mits een correcte implementatie, robuuste monitoring en een goed gefundeerde netwerkkieze.

