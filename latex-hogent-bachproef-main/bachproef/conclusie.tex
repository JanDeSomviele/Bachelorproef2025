%%=============================================================================
%% Conclusie
%%=============================================================================

\chapter{Conclusie}%
\label{ch:conclusie}

% TODO: Trek een duidelijke conclusie, in de vorm van een antwoord op de
% onderzoeksvra(a)g(en). Wat was jouw bijdrage aan het onderzoeksdomein en
% hoe biedt dit meerwaarde aan het vakgebied/doelgroep? 
% Reflecteer kritisch over het resultaat. In Engelse teksten wordt deze sectie
% ``Discussion'' genoemd. Had je deze uitkomst verwacht? Zijn er zaken die nog
% niet duidelijk zijn?
% Heeft het onderzoek geleid tot nieuwe vragen die uitnodigen tot verder 
%onderzoek?


\section{Conclusie}

\subsection{Antwoord op de onderzoeksvraag}

Deze bachelorproef had als doel om te onderzoeken welk draadloos netwerk — Wi-Fi, 4G of 5G — het meest geschikt is voor industriële communicatie in een testomgeving waarin Modbus-protocollen, lichtsturing en HTTP-verkeer via Node-RED centraal staan. Uit de uitgevoerde metingen en functionele testen blijkt dat elk netwerk verschillende sterktes vertoont, afhankelijk van de toepassing.

4G kwam naar voren als de meest stabiele en veelzijdige keuze voor toepassingen die betrouwbaarheid en continuïteit vereisen. Wi-Fi biedt lage latency bij algemene netwerkcommunicatie, maar toonde zich minder stabiel bij functionele opdrachten zoals lichtsturing. 5G behaalt uitzonderlijk hoge bandbreedte en snelle reacties bij functionele testen, maar kende ook grotere schommelingen in latency en jitter. Dit maakt 5G geschikt voor toepassingen met hoge datasnelheden, op voorwaarde dat de netwerkconfiguratie (bijvoorbeeld privaat 5G) een zekere voorspelbaarheid kan bieden.

\subsection{Bijdrage en meerwaarde voor het vakgebied}

De bijdrage van dit onderzoek ligt in het feit dat het niet alleen theoretische snelheden en latencies vergelijkt, maar die direct koppelt aan praktische industriële use-cases. Door de combinatie van meetgegevens (\texttt{iperf3}, \texttt{ping}, packet captures) en functionele testen (lichtsturing, HTTP-verkeer) levert dit onderzoek bruikbare inzichten voor bedrijven en integratoren binnen industriële automatisering, domotica en smart industry.

In tegenstelling tot veel theoretische studies biedt dit werk een praktisch, reproduceerbaar scenario dat dicht aanleunt bij echte implementaties. De gebruikte methodologie — gestandaardiseerde metingen gecombineerd met testcases — vormt een nuttige basis voor toekomstig onderzoek of technische beslissingsondersteuning in het veld.

\subsection{Kritische reflectie en interpretatie van de resultaten}

In grote lijnen bevestigen de resultaten de verwachtingen: 5G biedt de hoogste prestaties op vlak van doorvoersnelheid, 4G is zeer betrouwbaar en Wi-Fi is snel maar gevoeliger voor verstoringen. Toch waren er ook verrassingen. Zo vertoonde 5G onverwacht hoge jitter- en latencypieken, wat wijst op onvoorspelbaarheid in publieke netwerken. Anderzijds was de reactietijd van 5G bij lichtsturing de snelste van de drie netwerken, wat aangeeft dat het voor lokale, kortstondige opdrachten toch goed presteert.

Ook viel op dat Wi-Fi, ondanks de lage gemiddelde pingtijd, de traagste was bij reële opdrachten. Dit toont aan dat latency niet alles zegt, en dat andere netwerkkenmerken zoals beschikbaarheid, interferentie en lokale belasting mee bepalen hoe een netwerk in de praktijk aanvoelt.

\subsection{Beperkingen en nieuwe onderzoeksvragen}

Een beperking van deze studie is dat de 5G-verbinding een publieke variant betrof. De prestaties van private 5G-netwerken — met gegarandeerde bandbreedte, QoS en lage latency — zouden in een vervolgstudie kunnen worden onderzocht om een vollediger beeld te vormen. Daarnaast werd Modbus-verkeer slechts beperkt geanalyseerd, omdat een gedetailleerde interpretatie van de .pcap-bestanden meer tijd vergde dan voorzien. Een verder onderzoek naar protocolgedrag (Modbus RTU/TCP) over verschillende netwerken kan hierin aanvullende inzichten bieden.

Bovendien werd de latency enkel gemeten naar een extern IP-adres (8.8.8.8). Toekomstig onderzoek zou baat hebben bij tests naar een lokaal IP-adres binnen hetzelfde netwerk, om externe factoren zoals internetroutering of providervertraging uit te sluiten.

Ten slotte roept dit onderzoek nieuwe vragen op over hoe draadloze netwerken zich gedragen onder zwaardere belasting, over langere periodes en met meerdere gelijktijdige apparaten. De impact van roaming, interferentie, congestie en beveiligingsconfiguraties zijn relevante vervolgvragen die verder onderzocht kunnen worden.

\subsection{Slotbeschouwing}

Deze bachelorproef bevestigt dat de keuze van een draadloos netwerk sterk afhangt van de specifieke vereisten van een industriële toepassing. Wi-Fi is snel maar kwetsbaar, 4G is stabiel en betrouwbaar, en 5G biedt het meeste potentieel voor geavanceerde toepassingen mits juiste implementatie. Door dit alles in kaart te brengen binnen een gecontroleerde testomgeving, levert dit werk waardevolle inzichten voor ingenieurs en besluitvormers in het domein van industriële connectiviteit.


