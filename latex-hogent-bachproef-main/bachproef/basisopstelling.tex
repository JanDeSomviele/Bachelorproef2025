\chapter{\IfLanguageName{dutch}{basisopstelling}{}}%
\label{ch:basisopstelling}

Om snel van start te kunnen gaan met het uitvoeren van testen en metingen werd een basisopstelling uitgewerkt waarbij geen gebruik wordt gemaakt van een externe server. Deze opstelling dient als eerste implementatie van het testnetwerk en biedt voldoende functionaliteit om al cruciale netwerkanalyses en communicatieproeven uit te voeren. In dit hoofdstuk wordt deze basisopstelling besproken, samen met de gebruikte componenten, verbindingen en de functionele doelen die hiermee gerealiseerd worden.

\section*{Doel van de basisopstelling}

De bedoeling van deze initiële configuratie is om een minimale maar representatieve omgeving op te zetten waarin het gedrag van een gebouwbeheersysteem (GBS) getest kan worden over zowel een privaat 4G- als 5G-netwerk. De nadruk ligt hierbij op de communicatie tussen een Spacelogic AS-P controller en één of meerdere TAC Xenta-modules. Deze componenten vormen een realistische representatie van HVAC- en verlichtingssystemen in gebouwen.

Door het weglaten van een centrale server worden de scripts en logging lokaal uitgevoerd op een laptop of test-PC die direct met het netwerk verbonden is. Dit verlaagt de complexiteit van de opstartfase en maakt het eenvoudiger om foutopsporing te doen tijdens de eerste tests.

\section*{Componenten in de opstelling}

De basisopstelling bevat de volgende kerncomponenten:
\begin{itemize}
    \item \textbf{Spacelogic AS-P Controller}: functioneert als centrale regelaar die via Modbus TCP communiceert met de Xenta-modules.
    \item \textbf{TAC Xenta Module (type 401)}: simulatie van een HVAC- of verlichtingsmodule die uitleesbaar is via Modbus.
    \item \textbf{Laptop of test-PC}: voert de testscripts uit en verzamelt meetgegevens (zoals latency, responstijden, packet loss, etc.).
    \item \textbf{Mobiel netwerk (4G of 5G)}: verbinding tussen de PC en de controller gebeurt via een mobiele router of gateway. Er wordt gebruikgemaakt van het privaat 4G/5G-netwerk beschikbaar op de HOGENT-campus.
\end{itemize}

\section*{Verbindingsstructuur}
% TO DO: foto/de structuur kunnen tonen met dan uitleg eronder
De verbindingen worden als volgt gelegd:
\begin{enumerate}
    \item De AS-P controller wordt verbonden met een mobiele router/gateway via ethernet.
    \item De test-PC/laptop wordt via USB of ethernet verbonden met dezelfde mobiele router.
    \item De mobiele router is verbonden met het 4G- of 5G-netwerk, afhankelijk van het scenario.
    \item De TAC Xenta-module is bekabeld aangesloten op de AS-P controller via een RS-485-bus of IP, afhankelijk van de configuratie.
    \item Er is geen centrale server, dus alle scripts draaien op de test-PC.
\end{enumerate}

Deze opstelling maakt het mogelijk om onder reële netwerkcondities de communicatie en responstijd van het GBS te meten en zo de betrouwbaarheid van beide netwerken te evalueren.

\section*{Functionele werking van de opstelling}

De communicatie tussen de PC en de controller wordt voortdurend gemonitord door de Python- en Bash-scripts die eerder besproken werden. Zo worden de volgende metingen uitgevoerd:
\begin{itemize}
    \item Latency tussen PC en controller via ping-tests.
    \item Bandbreedte en jitter via iperf3 indien ondersteund door de router.
    \item Realtime uitlezing van registers via Modbus TCP.
    \item Registratie van eventuele time-outs of fouten in de communicatie.
\end{itemize}

Alle meetresultaten worden gelogd en gestructureerd opgeslagen om later geanalyseerd te worden in functie van het gekozen netwerktype (4G of 5G).

\section*{Voordelen van deze opstelling}
% TO DO: doorlopende tekst
\begin{itemize}
    \item \textbf{Eenvoudige opstart}: geen nood aan serverinstallatie of configuratie.
    \item \textbf{Directe controle en logging}: alles gebeurt lokaal op de test-PC.
    \item \textbf{Duidelijke afbakening van testdomein}: beperkt aantal variabelen maakt het gemakkelijker om netwerkinvloed te isoleren.
    \item \textbf{Schaalbaar}: indien later nodig, kan een server eenvoudig worden toegevoegd aan deze basisstructuur.
\end{itemize}