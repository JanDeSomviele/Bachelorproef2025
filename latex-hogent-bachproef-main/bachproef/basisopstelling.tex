\chapter{\IfLanguageName{dutch}{De opstelling}{The set-up}}%
\label{ch:basisopstelling}

De opstelling is ontworpen om zo snel mogelijk connectiviteitstesten en netwerkmetingen uit te voeren. De oorspronkelijke bedoeling is om gebruik te maken van een Schneider Electric SpaceLogic SmartX AS-P controller als centrale eenheid voor de aansturing van de HVAC- en lichtsystemen. Omdat dit toestel niet onmiddellijk beschikbaar is en de focus in eerste instantie ligt op het testen van de netwerkinfrastructuur eerder dan op de volledige gebouwbeheersoftware, wordt gekozen voor een Raspberry Pi als vervangend testplatform.

De Raspberry Pi fungeert in deze opstelling als een centrale node die zowel scripts uitvoert om netwerkmetingen te doen als eenvoudige Modbus-polling uitvoert naar gesimuleerde of echte apparaten. De keuze voor de Raspberry Pi biedt een aantal voordelen:

\begin{itemize}
    \item Flexibiliteit: De Raspberry Pi ondersteunt Python, bash en diverse bibliotheken zoals `pymodbus`, `speedtest-cli` en `psutil`, waarmee netwerktesten eenvoudig geautomatiseerd worden.
    \item Lage kostprijs en eenvoudige setup: Dankzij de brede community en ondersteuning kan men snel aan de slag.
    \item Mobiliteit: Door het compacte formaat wordt de Pi ook ingezet als mobiele testnode binnen verschillende netwerkzones (WiFi, 4G, 5G).
\end{itemize}

De opstelling bevat geen echte HVAC- of lichtsystemen, maar maakt gebruik van gesimuleerde Modbus-slave software. Deze wordt lokaal of op een andere computer uitgevoerd en reageert op pollingverzoeken van de Raspberry Pi. Dit maakt het mogelijk om latenties, foutpercentages en betrouwbaarheid van het netwerk te meten zonder dat de fysieke installaties al aanwezig moeten zijn.

Scripts worden eerst getest op een lokaal WiFi-netwerk, voordat ze ingezet worden binnen het private 5G-netwerk op de campus van HOGENT of een publiek netwerk. Zo wordt in een gecontroleerde omgeving de betrouwbaarheid en correcte werking van de scripts geverifieerd.

\section{Schema van de opstelling}
%
%\begin{figure}[H]
%    \centering
%    \includegraphics[width=0.95\linewidth]{images/basisopstelling_raspi.png}
%    \caption{Schematische voorstelling van de opstelling met Raspberry Pi}
%\end{figure}

\textbf{Beschrijving van het schema:}
\begin{itemize}
    \item De Raspberry Pi is via WiFi verbonden met het lokale netwerk (eigen router of access point). Dit netwerk fungeert als testomgeving voor netwerkmetingen.
    \item Via Python-scripts op de Pi wordt een Modbus TCP-client uitgevoerd, die connecteert met een Modbus TCP-server (slave simulator) op een andere computer of lokaal op de Pi.
    \item Tegelijk worden metingen uitgevoerd zoals ping, packet loss, jitter en bandbreedte, met behulp van netwerkscripts.
    \item De verzamelde data wordt lokaal opgeslagen in logbestanden of eenvoudige CSV-bestanden.
\end{itemize}

Deze opstelling vormt het fundament waarop latere uitbreidingen mogelijk zijn, zoals een echte 5G-verbinding, fysieke controllers, sensoren of een serveromgeving voor geavanceerde dataverwerking.

