\chapter{\IfLanguageName{dutch}{resultaten}{results}}%
\label{ch:resultaten}

In dit hoofdstuk worden de resultaten gepresenteerd van de uitgevoerde netwerk- en systeemtesten, zoals beschreven in het methodologiehoofdstuk. De metingen vinden plaats in 3 verschillende netwerkomgevingen: Wi-Fi (als referentie), 4G en privaat 5G. Voor elke omgeving worden gegevens verzameld over latency, jitter, packet loss, bandbreedte en applicatieresponsiviteit.

De resultaten worden gegroepeerd per netwerktype en gevisualiseerd met behulp van grafieken en tabellen. Naast objectieve meetwaarden wordt ook de functionele impact op de communicatie met HVAC- en verlichtingssimulaties besproken.

\section{Netwerkprestaties}

De netwerkprestaties worden vergeleken op basis van:

\begin{itemize}
    \item \textbf{Gemiddelde latency} (ms): de tijd tussen het verzenden en ontvangen van datapakketten;
    \item \textbf{Jitter} (ms): de variatie in vertragingstijd tussen opeenvolgende pakketten;
    \item \textbf{Packet loss} (\%): het percentage datapakketten dat verloren gaat tijdens verzending;
    \item \textbf{Bandbreedte} (Mbit/s): de gemeten download- en uploadsnelheden.
\end{itemize}

\subsection{Latency}

\subsection{Jitter}

\subsection{Packet loss}

Uit de ping test is er geen packet loss bij alle 3 de netwerken. Bij de iperf3 udp test ook geen data loss.

\subsection{Bandbreedte}



\textit{(Hier worden grafieken of tabellen ingevoegd, met interpretaties. Bijvoorbeeld: “In figuur 4.2 blijkt dat privaat 5G gemiddeld 20 ms lagere latency vertoont dan 4G onder gelijkaardige omstandigheden.”)}

\section{Functionele testen}

De reactietijden van Modbus-communicatie en het schakelen van verlichting worden actief gemeten. Daarbij wordt geanalyseerd of netwerkvertragingen een merkbare invloed uitoefenen op de betrouwbaarheid van de data-uitwisseling.

\begin{itemize}
    \item \textbf{Modbus polling}: er treden geen merkbare fouten op bij Wi-Fi of privaat 5G. Bij 4G worden sporadisch time-outs gemeten onder belasting;
    \item \textbf{Verlichting}: bij publiek 5G doen zich occasioneel merkbare vertragingen voor van 1 tot 2 seconden tijdens intensieve belasting.
\end{itemize}

\section{Observaties tijdens simulaties}

Tijdens simulaties van netwerklast en variërende signaalsterkte worden volgende observaties genoteerd:




