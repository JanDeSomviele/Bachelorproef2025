\chapter{\IfLanguageName{dutch}{resultaten}{results}}%
\label{ch:resultaten}

%TO DO: uitschrijven

In dit hoofdstuk worden de resultaten gepresenteerd van de uitgevoerde netwerk- en systeemtesten, zoals beschreven in het methodologiehoofdstuk. De metingen vinden plaats in 3 verschillende netwerkomgevingen: Wi-Fi (als referentie), 4G en privaat 5G. Voor elke omgeving worden gegevens verzameld over latency, jitter, packet loss, bandbreedte en applicatieresponsiviteit.

De resultaten worden gegroepeerd per netwerktype en gevisualiseerd met behulp van grafieken en tabellen. Naast objectieve meetwaarden wordt ook de functionele impact op de communicatie met HVAC- en verlichtingssimulaties besproken.

\section{Netwerkprestaties}

De netwerkprestaties worden vergeleken op basis van:

\begin{itemize}
    \item \textbf{Gemiddelde latency} (ms): de tijd tussen het verzenden en ontvangen van datapakketten;
    \item \textbf{Jitter} (ms): de variatie in vertragingstijd tussen opeenvolgende pakketten;
    \item \textbf{Packet loss} (\%): het percentage datapakketten dat verloren gaat tijdens verzending;
    \item \textbf{Bandbreedte} (Mbit/s): de gemeten download- en uploadsnelheden.
\end{itemize}

\subsection{Latency}

\subsection{Jitter}

\subsection{Packet loss}

Uit de ping test is er geen packet loss bij alle 3 de netwerken. Bij de iperf3 udp test ook geen data loss.

\subsection{Bandbreedte}



\textit{(Hier worden grafieken of tabellen ingevoegd, met interpretaties. Bijvoorbeeld: “In figuur 4.2 blijkt dat privaat 5G gemiddeld 20 ms lagere latency vertoont dan 4G onder gelijkaardige omstandigheden.”)}

\section{Functionele testen}

De reactietijden van Modbus-communicatie en het schakelen van verlichting worden actief gemeten. Daarbij wordt geanalyseerd of netwerkvertragingen een merkbare invloed uitoefenen op de betrouwbaarheid van de data-uitwisseling.

\begin{itemize}
    \item \textbf{Modbus polling}: 
    \item \textbf{Verlichting}: 
\end{itemize}
\subsection{Modbus polling}
%TO DO: tcpdump analyseren (zie. chatgpt)
% Hoe Modbus polling afleiden uit een .pcap
%1. Open het .pcap-bestand in Wireshark
%Gebruik filters om enkel Modbus TCP-verkeer te bekijken:
%
%tcp.port == 502
%Of specifiek voor Modbus:
%
%modbus
%2. Zoek naar Modbus function codes
%Voor polling zijn dit meestal:
%
%Function Code 03 – Read Holding Registers
%
%Function Code 04 – Read Input Registers
%
%Klik op een request en bekijk de tijdstempel (in de eerste kolom of via Frame > Arrival Time).
%Zoek daarna de bijbehorende response-packet.
%
%3. Bereken de polling latency
%Trek de tijd van het verzoek af van de tijd van het antwoord:
%
%Latency = T(response) - T(request)
%Bijvoorbeeld:
%
%makefile
%Request: 10:15:02.341
%Response: 10:15:02.397
%→ Modbus latency ≈ 56 ms
%Je kunt dit handmatig doen voor een representatieve steekproef (bv. 10 requests per netwerk), of je kunt exporteren naar CSV en automatisch laten verwerken via Python.
%
% Hoe dit vermelden in je verslag
%Functionele testen > Modbus polling:
%\item \textbf{Modbus polling (afgeleid via Wireshark)}: In elk netwerk is een .pcap-bestand geanalyseerd in Wireshark. Daarbij wordt gekeken naar de tijd tussen Modbus TCP-verzoeken (bijv. “Read Holding Registers”) en hun bijhorende antwoorden. Uit deze analyse blijkt dat de gemiddelde polling-responstijd onder Wi-Fi stabiel blijft rond XX ms, terwijl bij 4G en 5G lichte schommelingen optreden met maxima tot XX ms. Deze waarden zijn consistent met de gemeten netwerkvertragingen.
%Laat me gerust weten of je hulp wil met:
%
%het filteren en interpreteren van specifieke pcap-sessies,

\subsection{Verlichting}

\section{Observaties tijdens simulaties}

Tijdens simulaties van netwerklast en variërende signaalsterkte worden volgende observaties genoteerd:




