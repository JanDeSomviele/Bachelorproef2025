%==============================================================================
% Sjabloon onderzoeksvoorstel bachproef
%==============================================================================
% Gebaseerd op document class `hogent-article'
% zie <https://github.com/HoGentTIN/latex-hogent-article>

% Voor een voorstel in het Engels: voeg de documentclass-optie [english] toe.
% Let op: kan enkel na toestemming van de bachelorproefcoördinator!
\documentclass{hogent-article}

% Invoegen bibliografiebestand
\addbibresource{voorstel.bib}

% Informatie over de opleiding, het vak en soort opdracht
\studyprogramme{Professionele bachelor toegepaste informatica}
\course{Bachelorproef}
\assignmenttype{Onderzoeksvoorstel}
% Voor een voorstel in het Engels, haal de volgende 3 regels uit commentaar
% \studyprogramme{Bachelor of applied information technology}
% \course{Bachelor thesis}
% \assignmenttype{Research proposal}

\academicyear{2024-2025} 

% TODO: Werktitel
\title{De impact van mobiele netwerken op facilitair beheer: een vergelijkende studie van 4G, privaat 5G en publiek 5G voor HOGENT}


\author{Jan De Somviele}
\email{jan.desomviele@student.hogent.be}


% TODO: Geef de co-promotor op
\supervisor[Co-promotor]{G. Vercauteren (HOGENT, \href{mailto:giselle.vercauteren@hogent.be}{giselle.vercauteren@hogent.be})}

% Binnen welke specialisatierichting uit 3TI situeert dit onderzoek zich?
% Kies uit deze lijst:
%
% - Mobile \& Enterprise development
% - AI \& Data Engineering
% - Functional \& Business Analysis
% - System \& Network Administrator
% - Mainframe Expert
% - Als het onderzoek niet past binnen een van deze domeinen specifieer je deze
%   zelf
%
\specialisation{System \& Network Administrator}
\keywords{4G, 5G, facilitair beheer}

\begin{document}

\begin{abstract}
    % TO DO: Hier schrijf je de samenvatting van je voorstel, als een doorlopende tekst van één paragraaf. Let op: dit is geen inleiding, maar een samenvattende tekst van heel je voorstel met inleiding (voorstelling, kaderen thema), probleemstelling en centrale onderzoeksvraag, onderzoeksdoelstelling (wat zie je als het concrete resultaat van je bachelorproef?), voorgestelde methodologie, verwachte resultaten en meerwaarde van dit onderzoek (wat heeft de doelgroep aan het resultaat?).
Deze bachelorproef onderzoekt de overstap van een bekabeld netwerk naar mobiele netwerken zoals 4G, privaat 5G en publiek 5G voor de facilitair diensten verwarming en verlichting van HOGENT. Doordat facilitaire apparatuur steeds meer afhankelijk is van internettoegang en cloudmanagement is het nodig om een netwerk te kiezen die een goede balans biedt tussen beveiliging, prestaties en netwerkbelasting. De centrale onderzoeksvraag van deze bachelorproef is: \textit{Welk netwerk (4G, privaat 5G of publiek 5G) biedt de beste balans tussen beveiliging, prestaties en netwerkbelasting voor de facilitaire diensten verwarming en verlichting van HOGENT?} De voorgestelde methodologie omvat een literatuurstudie, een vergelijkende analyse van netwerkmogelijkheden, simulaties en een proof-of-concept uitgevoerd op de 5G-omgeving van de campus Schoonmeersen. Het verwachte resultaat van het onderzoek is dat een privaat 5G netwerk de beste balans biedt en als deze niet beschikbaar is een combinatie van 4G en 5G een passend alternatief is. Het onderzoek resulteert in een rapport met concrete aanbevelingen voor HOGENT, waarmee de netwerkkeuzes voor verwarming en verlichting verbeterd kunnen worden. De meerwaarde van deze studie ligt in het leveren van inzichten die bijdragen aan efficiëntere, veiligere en toekomstbestendige netwerkoplossingen binnen HOGENT.
\end{abstract}

\tableofcontents

% De hoofdtekst van het voorstel zit in een apart bestand, zodat het makkelijk
% kan opgenomen worden in de bijlagen van de bachelorproef zelf.
%---------- Inleiding ---------------------------------------------------------

\section{Inleiding}%
\label{sec:inleiding}

%Waarover zal je bachelorproef gaan? Introduceer het thema en zorg dat volgende zaken zeker duidelijk aanwezig zijn:
%
%\begin{itemize}
%  \item kaderen thema
%  \item de doelgroep
%  \item de probleemstelling en (centrale) onderzoeksvraag
%  \item de onderzoeksdoelstelling
%\end{itemize}
%
%Denk er aan: een typische bachelorproef is \textit{toegepast onderzoek}, wat betekent dat je start vanuit een concrete probleemsituatie in bedrijfscontext, een \textbf{casus}. Het is belangrijk om je onderwerp goed af te bakenen: je gaat voor die \textit{ene specifieke probleemsituatie} op zoek naar een goede oplossing, op basis van de huidige kennis in het vakgebied.
%
%De doelgroep moet ook concreet en duidelijk zijn, dus geen algemene of vaag gedefinieerde groepen zoals \emph{bedrijven}, \emph{developers}, \emph{Vlamingen}, enz. Je richt je in elk geval op it-professionals, een bachelorproef is geen populariserende tekst. Eén specifiek bedrijf (die te maken hebben met een concrete probleemsituatie) is dus beter dan \emph{bedrijven} in het algemeen.
%
%Formuleer duidelijk de onderzoeksvraag! De begeleiders lezen nog steeds te veel voorstellen waarin we geen onderzoeksvraag terugvinden.


%Schrijf ook iets over de doelstelling. Wat zie je als het concrete eindresultaat van je onderzoek, naast de uitgeschreven scriptie? Is het een proof-of-concept, een rapport met aanbevelingen, \ldots Met welk eindresultaat kan je je bachelorproef als een succes beschouwen?

Deze bachelorproef richt zich op het onderzoeken van de impact van de overstap van een bekabeld netwerk naar mobiele netwerken, zoals 4G, privaat 5G, en publiek 5G, voor de facilitaire diensten verwarming en verlichting van HOGENT. Doordat apparatuur steeds meer gebruik maakt van internettoegang en cloudmanagement, wordt de vraag gesteld hoe dit met 4G en 5G kan opgelost worden. Voor HOGENT betekent dit een overstap van een bekabeld netwerk naar een mobiel netwerk. Een mobiel netwerk brengt zowel voordelen als uitdagingen mee: een mobiel netwerk belooft snellere verbindingen en flexibiliteit, maar zorgt ook voor de nood aan afwegingen met betrekking tot bandbreedte, netwerkbelasting en veiligheid. De doelgroep van dit onderzoek is het faciliteir beheer van HOGENT. De centrale onderzoeksvraag van deze bachelorproef is: \textit{Welk netwerk (4G, privaat 5G of publiek 5G) biedt de beste balans tussen beveiliging, prestaties en netwerkbelasting voor de facilitaire diensten verwarming en verlichting van HOGENT?}. Naast deze onderzoeksvraag worden volgende vragen ook beantwoord: 
\begin{itemize}
    \item Welke technische vereisten hebben de systemen voor verwarming en verlichting op HOGENT-locaties?
    \item Wat zijn de veiligheidsrisico’s bij het gebruik van mobiele netwerken voor verwarming en verlichting, en hoe kunnen deze worden beperkt?
    \item Welke aanpassingen zijn nodig om de bestaande bekabelde systemen voor verwarming en verlichting compatibel te maken met mobiele netwerken?
    \item Wat zijn de implicaties van de keuze voor een mobiel netwerk op de operationele continuïteit van de diensten verwarming en verlichting?
    \item Welke eisen stellen de apparaten en applicaties voor verwarming en verlichting op dit moment aan het netwerk?
    \item Wat zijn de implicaties van een overstap naar een mobiel netwerk voor onderhoud en beheer van de verwarmings- en verlichtingssystemen?
    \item Is het zinvol om volledig over te stappen naar een privaat 5G netwerk, en hoe kan dit het best worden gerealiseerd?
    \item Welke netwerkmogelijkheden zijn beschikbaar op HOGENT-locaties die niet over privaat 5G zullen beschikken, en wat zijn de aanbevolen oplossingen voor dergelijke locaties?
\end{itemize}

%---------- Stand van zaken ---------------------------------------------------

\section{Literatuurstudie}%
\label{sec:literatuurstudie}
%
%Hier beschrijf je de \emph{state-of-the-art} rondom je gekozen onderzoeksdomein, d.w.z.\ een inleidende, doorlopende tekst over het onderzoeksdomein van je bachelorproef. Je steunt daarbij heel sterk op de professionele \emph{vakliteratuur}, en niet zozeer op populariserende teksten voor een breed publiek. Wat is de huidige stand van zaken in dit domein, en wat zijn nog eventuele open vragen (die misschien de aanleiding waren tot je onderzoeksvraag!)?
%
%Je mag de titel van deze sectie ook aanpassen (literatuurstudie, stand van zaken, enz.). Zijn er al gelijkaardige onderzoeken gevoerd? Wat concluderen ze? Wat is het verschil met jouw onderzoek?
%
%Verwijs bij elke introductie van een term of bewering over het domein naar de vakliteratuur, bijvoorbeeld~\autocite{Hykes2013}! Denk zeker goed na welke werken je refereert en waarom.
%
%Draag zorg voor correcte literatuurverwijzingen! Een bronvermelding hoort thuis \emph{binnen} de zin waar je je op die bron baseert, dus niet er buiten! Maak meteen een verwijzing als je gebruik maakt van een bron. Doe dit dus \emph{niet} aan het einde van een lange paragraaf. Baseer nooit teveel aansluitende tekst op eenzelfde bron.
%
%Als je informatie over bronnen verzamelt in JabRef, zorg er dan voor dat alle nodige info aanwezig is om de bron terug te vinden (zoals uitvoerig besproken in de lessen Research Methods).
%
%% Voor literatuurverwijzingen zijn er twee belangrijke commando's:
%% \autocite{KEY} => (Auteur, jaartal) Gebruik dit als de naam van de auteur
%%   geen onderdeel is van de zin.
%% \textcite{KEY} => Auteur (jaartal)  Gebruik dit als de auteursnaam wel een
%%   functie heeft in de zin (bv. ``Uit onderzoek door Doll & Hill (1954) bleek
%%   ...'')
%
%Je mag deze sectie nog verder onderverdelen in subsecties als dit de structuur van de tekst kan verduidelijken.


\subsection{Facilitair beheer}
Facilitair beheer (FM) is een interdisciplinair vakgebied dat de coördinatie van mensen, processen en ruimtes omvat om welzijn en efficiëntie binnen organisaties te verbeteren (ISO 41011:2018) \autocite{jaouhari2023we}. 

\subsection{Vergelijking 4G en 5G}
Het verschil tussen 4G en 5G in facilitair beheer ligt in de infrastructuur en middelenbeheer. 5G biedt flexibelere en efficiëntere infrastructuur dankzij virtualisatie wat zorgt voor schaalbaarheid en minder handmatige interventie \autocite{degambur2021resource}. 5G introduceert een verbeterde netwerkcapaciteit, betrouwbaarheid en efficiëntie, met lagere latentie en lager energieverbruik ten opzichte van 4G. De nadruk ligt op hoge snelheden en het verbinden van meerdere apparaten tegelijkertijd \autocite{mihret20214g}. De integratie van 5G technologie in het facilitair beheer van slimme gebouwen biedt voordelen voor efficiëntie, connectiviteit en veiligheid. Hoewel uitdagingen zoals hoge investeringskosten en complexe integratie bestaan, maar de voordelen van verbeterde duurzaamheid, efficiëntie en comfort overtreffen deze uitdagingen \autocite{Markogiannaki2023}. Private 5G netwerken bieden bedrijven en organisaties een veilig, flexibel en schaalbaar alternatief voor publieke 5G netwerken. Private 5G netwerken maken het mogelijk om data veilig en autonoom te delen zonder de openbare internetinfrastructuur te gebruiken \autocite{eswaran2023private}.






%---------- Methodologie ------------------------------------------------------
\section{Methodologie}
\label{sec:methodologie}

%Hier beschrijf je hoe je van plan bent het onderzoek te voeren. Welke onderzoekstechniek ga je toepassen om elk van je onderzoeksvragen te beantwoorden? Gebruik je hiervoor literatuurstudie, interviews met belanghebbenden (bv.~voor requirements-analyse), experimenten, simulaties, vergelijkende studie, risico-analyse, PoC, \ldots?
%
%Valt je onderwerp onder één van de typische soorten bachelorproeven die besproken zijn in de lessen Research Methods (bv.\ vergelijkende studie of risico-analyse)? Zorg er dan ook voor dat we duidelijk de verschillende stappen terug vinden die we verwachten in dit soort onderzoek!
%
%Vermijd onderzoekstechnieken die geen objectieve, meetbare resultaten kunnen opleveren. Enquêtes, bijvoorbeeld, zijn voor een bachelorproef informatica meestal \textbf{niet geschikt}. De antwoorden zijn eerder meningen dan feiten en in de praktijk blijkt het ook bijzonder moeilijk om voldoende respondenten te vinden. Studenten die een enquête willen voeren, hebben meestal ook geen goede definitie van de populatie, waardoor ook niet kan aangetoond worden dat eventuele resultaten representatief zijn.
%
%Uit dit onderdeel moet duidelijk naar voor komen dat je bachelorproef ook technisch voldoen\-de diepgang zal bevatten. Het zou niet kloppen als een bachelorproef informatica ook door bv.\ een student marketing zou kunnen uitgevoerd worden.
%
%Je beschrijft ook al welke tools (hardware, software, diensten, \ldots) je denkt hiervoor te gebruiken of te ontwikkelen.
%
%Probeer ook een tijdschatting te maken. Hoe lang zal je met elke fase van je onderzoek bezig zijn en wat zijn de concrete \emph{deliverables} in elke fase?


\subsection{Literatuuronderzoek en probleemdefinitie}
Het doel van deze fase is om inzicht te krijgen in de huidige situatie en de technische vereisten van de systemen voor verwarming en verlichting op HOGENT-locaties. Hiervoor wordt een literatuurstudie uitgevoerd om de technische mogelijkheden en beperkingen van 4G, privaat 5G en publiek 5G te analyseren voor verwarming en verlichting. Ook worden veiligheidsrisico's bij het gebruik van mobiele netwerken onderzocht en hoe deze kunnen vermeden worden. Daarnaast worden gesprekken gevoerd met belanghebbenden bij HOGENT, zoals technisch personeel en de IT-afdeling, om een requirementsanalyse op te stellen die inzicht biedt in de huidige netwerkvereisten en mogelijke aanpassingen voor de bestaande systemen. De data voor deze fase wordt verzameld uit academische literatuur via Google Scholar, technische documentatie van HOGENT en input van interne belanghebbenden. De deliverable is een rapport waarin de technische vereisten en veiligheidsrisico’s worden gedefinieerd, en de benodigde aanpassingen om de huidige systemen compatibel te maken met mobiele netwerken.

\subsection{Vergelijkende studie van netwerktechnologieën}
In deze fase wordt een vergelijkende studie uitgevoerd om te beoordelen welke netwerktechnologie het meest geschikt is voor de diensten verwarming en verlichting op HOGENT. Dit omvat een evaluatie van de technische eisen die de apparaten en applicaties momenteel van het netwerk vragen. Daarnaast gebeurt een analyse van de gevolgen van een overstap naar een mobiel netwerk voor operationele continuïteit, onderhoud en beheer voor de gekozen toepassingen. Data voor deze fase wordt verzameld uit literatuur, simulaties met tools zoals Cisco Packet Tracer en experimentele data uit de 5G-omgeving van de campus Schoonmeersen. De deliverable is een rapport met een uitgebreide analyse van de geschiktheid van 4G, privaat 5G en publiek 5G, inclusief aanbevelingen voor locatiespecifieke netwerkmogelijkheden.

\subsection{Simulaties en proof of concept}
Deze fase richt zich op het valideren van de theoretische bevindingen door middel van simulaties en praktische tests. Netwerkprestaties worden gesimuleerd in scenario's zoals bijvoorbeeld het aansturen van slimme thermostaten en dynamische verlichting op verschillende campuslocaties. Hierbij wordt de operationele continuïteit van de diensten onderzocht en worden de vereisten voor onderhoud en beheer geëvalueerd. Daarnaast wordt een proof of concept uitgevoerd in de 5G-omgeving van de campus Schoonmeersen, waarbij IoT-apparaten voor verwarming en verlichting worden getest op compatibiliteit en prestaties in mobiele netwerken. De benodigde data omvat simulatiegegevens, resultaten van de proof of concept en specificaties van de gebruikte apparaten. De deliverable van deze fase is een analyse van de resultaten van de simulaties en de proof of concept voor de haalbaarheid en impact van de overstap naar mobiele netwerken.

\subsection{Data-analyse en aanbevelingen}
De laatste fase beantwoordt de overkoepelende onderzoeksvraag en de deelvragen. De verzamelde data uit de eerdere fasen wordt geanalyseerd om concrete aanbevelingen te formuleren over welke netwerktechnologie het meest geschikt is voor verwarming en verlichting op HOGENT. De deliverable van deze fase is een eindrapport met gedetailleerde aanbevelingen en implementatiestrategieën waarmee HOGENT een geïnformeerde beslissing kan nemen over de netwerkkeuzes voor haar facilitaire diensten.




%---------- Verwachte resultaten ----------------------------------------------
\section{Verwacht resultaat, conclusie}%
\label{sec:verwachte_resultaten}

%Hier beschrijf je welke resultaten je verwacht. Als je metingen en simulaties uitvoert, kan je hier al mock-ups maken van de grafieken samen met de verwachte conclusies. Benoem zeker al je assen en de onderdelen van de grafiek die je gaat gebruiken. Dit zorgt ervoor dat je concreet weet welk soort data je moet verzamelen en hoe je die moet meten.
%
%Wat heeft de doelgroep van je onderzoek aan het resultaat? Op welke manier zorgt jouw bachelorproef voor een meerwaarde?
%
%Hier beschrijf je wat je verwacht uit je onderzoek, met de motivatie waarom. Het is \textbf{niet} erg indien uit je onderzoek andere resultaten en conclusies vloeien dan dat je hier beschrijft: het is dan juist interessant om te onderzoeken waarom jouw hypothesen niet overeenkomen met de resultaten.

\subsection{Netwerkbelasting en bandbreedte}
Er wordt verwacht dat 5G netwerken een lagere netwerkbelasting en hogere bandbreedte kunnen bieden voor facilitaire diensten zoals verwarming en verlichting. 

\subsection{Beveiligingsimpact}
Het onderzoek verwacht dat een privaat 5G-netwerk een hoger niveau van beveiliging kan bieden voor de facilitaire diensten verwarming en verlichting dan publiek 5G, aangezien private netwerken exclusief toegankelijk zijn voor interne facilitaire diensten. Er wordt verwacht dat bij 4G het niveau van beveilging nog lager ligt.

\subsection{Meerwaarde voor de doelgroep}
De meerwaarde voor de doelgroep van deze bachelorproef is om een inzicht te geven in welk netwerktype het meest geschikt is voor de facilitair diensten verwarming en verlichting voor HOGENT. 

\subsection{Verwachte conclusie}
De conclusie van de verwachte resultaten is dat een privaat 5G-netwerk de beste balans biedt tussen beveiliging en prestaties voor de verwarming en verlichting van HOGENT. Als er geen toegang is aan een privaat 5G netwerk dan kan een combinatie van 4G en publiek 5G een alternatief zijn.



\printbibliography[heading=bibintoc]

\end{document}