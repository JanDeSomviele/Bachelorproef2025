%==============================================================================
% Sjabloon poster bachproef
%==============================================================================
% Gebaseerd op document class `a0poster' door Gerlinde Kettl en Matthias Weiser
% Aangepast voor gebruik aan HOGENT door Jens Buysse en Bert Van Vreckem

\documentclass[a0,portrait]{hogent-poster}

% Info over de opleiding
\course{Bachelorproef}
\studyprogramme{toegepaste informatica}
\academicyear{2024-2025}
\institution{Hogeschool Gent, Valentin Vaerwyckweg 1, 9000 Gent}

% Info over de bachelorproef
\title{De impact van mobiele netwerken op facilitair beheer: een vergelijkende studie van 4G en privaat 5G voor HOGENT}
%\subtitle{Ondertitel (eventueel)}
\author{Jan De Somviele}
\email{jan.desomviele@student.hogent.be}
\supervisor{Lena De Mol, Chantal Teerlinck}
\cosupervisor{Giselle Vercauteren (HOGENT)}

% Indien ingevuld, wordt deze informatie toegevoegd aan het einde van de
% abstract. Zet in commentaar als je dit niet wilt.
\specialisation{Systeem- en Netwerkbeheer}
\keywords{5G, 4G, gebouwbeheer}
\projectrepo{https://github.com/JanDeSomviele/Bachelorproef2025}

\begin{document}

\maketitle

\begin{abstract}
Deze poster onderzoekt de prestaties en betrouwbaarheid van 4G- en privaat 5G-netwerken in de context van gebouwbeheersystemen zoals verlichting en HVAC. Via netwerkmetingen en functionele testen worden de praktische implicaties en geschiktheid van mobiele netwerken binnen smart buildings geëvalueerd.
\end{abstract}

\begin{multicols}{2} % This is how many columns your poster will be broken into, a portrait poster is generally split into 2 columns

\section{Introductie}

In deze bachelorproef werd onderzocht hoe 4G en privaat 5G-netwerken presteren binnen een gebouwbeheersysteem (GBS), met toepassingen zoals verlichting en HVAC. De centrale onderzoeksvraag luidde: \textit{Wat is het verschil tussen 4G en privaat 5G in verband met prestaties en betrouwbaarheid van toepassingen binnen een GBS?}

\begin{center}
    \captionsetup{type=figure}
    \includegraphics[width=0.45\textwidth]{../graphics/beideOpstellingen.jpg}
    \captionof{figure}{Opstelling A: , Opstelling B:}
\end{center}
De opstelling heeft twee configuraties: beide hebben een pc bekabeld verbonden met een RUTX50-router die wisselend wifi, 4G of privaat 5G levert. Opstelling A heeft een Raspberry Pi die fungeert als meetnode en in opstelling B wordt deze vervangen door een Philips Hue Bridge en een Philips Hue smartlight.
\section{Experimenten}

\begin{center}
    \captionsetup{type=figure}
    \includegraphics[width=0.45\textwidth]{../graphics/Latency-Jitter-grafiek.png}
    \captionof{figure}{Grafiek 1: , Grafiek 2:}
\end{center}

\begin{multicols}{2}
\subsection*{Latency}

Latency is de tijd die een datapakket nodig heeft om heen en terug te reizen tussen twee punten, gemeten als round-trip time (RTT). In deze test werd met het commando 'ping -c 250' naar een extern IP gekeken naar de snelheid en stabiliteit van de netwerken. Wi-Fi had de laagste gemiddelde latency (13,27 ms), gevolgd door 4G (18,17 ms), terwijl 5G ondanks een lage minimumwaarde (10,85 ms) last had van grote pieken tot 230,92 ms. Dit kan wijzen op fluctuaties in de testomgeving of instabiliteit van het privaat 5G-netwerk.
\subsection*{Jitter}

Jitter geeft de variatie in vertraging tussen opeenvolgende datapakketten weer. Dit is belangrijk voor betrouwbare real-time toepassingen zoals HVAC en lichtsturing. Met iperf3 werden drie UDP-tests uitgevoerd per netwerk (1, 5 en 50 Mbit/s), waarbij jitter werd gemeten. Bij 1 Mbit/s had wifi de hoogste jitter (0,799 ms), terwijl 4G en 5G stabieler waren. Bij hogere snelheden daalden de jitterwaarden en stabiliseerden alle netwerken rond 0,15–0,20 ms.
\end{multicols}

\subsection*{Packet loss}
Packet loss werd getest met zowel de latency ping- als de jitter iperf3-test en bleek in alle netwerken nul procent te bedragen. Dit wijst op een stabiele en betrouwbare dataverbinding, zelfs bij hoge verzendbelasting. Voor kritieke toepassingen wordt aangeraden om packet loss ook onder realistischere omstandigheden verder te evalueren.

\begin{center}
    \captionsetup{type=figure}
    \includegraphics[width=0.45\textwidth]{../graphics/bandbreedteLicht.png}
    \captionof{figure}{Tabel 1: bandbreedte , Tabel 2: licht}
\end{center}

\begin{multicols}{2}
\subsection*{Bandbreedte}
De iperf3 TCP throughput-test meet hoeveel data per seconde succesvol wordt verzonden tussen de Rasberry Pi en de pc. Wifi en 4G presteren beide rond de 94 Mbit/s. 5G behaalt een veel hogere gemiddelde bitrate van 931 Mbit/s. Dit toont de grotere datacapaciteit van 5G. 

\subsection*{licht}
Bij deze test werd opstelling B gebruikt. De reactietijd tussen het versturen van een commando en het schakelen van een slimme lamp over wifi, 4G en 5G werd gemeten. Alle netwerken behaalden 100\% succesratio, maar 5G gaf de snelste reactie (gem. 90,63 ms), gevolgd door 4G en wifi. 
\end{multicols}
\begin{center}
    \captionsetup{type=figure}
    \includegraphics[width=0.45\textwidth]{../graphics/http.png}
    \captionof{figure}{Grafiek 1: spreidingsgrafiek download speed , Grafiek 2: gem snelheiden van http test}
\end{center}
\subsection*{http}
De Node-RED test simuleerde periodiek HTTP-verkeer om de prestaties van wifi, 4G en 5G te vergelijken bij IoT-toepassingen. Wifi had de laagste doorvoersnelheid en meer variatie, terwijl 4G het snelst en meest consistent presteerde. 5G scoorde ook goed, maar toonde iets meer schommelingen. Alle netwerken waren echter voldoende snel en stabiel voor standaard IoT-gebruik zoals statusupdates en eenvoudige opdrachten.


\section{Conclusies}
Uit de testen blijkt dat 4G de meest stabiele en betrouwbare prestaties levert, terwijl 5G uitblinkt in snelheid maar gevoelig is voor schommelingen in latency en jitter. Wifi scoorde goed qua latency, maar bleek minder betrouwbaar bij functionele opdrachten. 5G toont potentieel voor veeleisende toepassingen, mits gebruik in een gecontroleerde (private) omgeving.

\section{Toekomstig onderzoek}

Toekomstig onderzoek kan zich richt op testen binnen een private 5G-omgeving en langdurige metingen onder reële omstandigheden. Daarbij wordt gekeken naar schaalbaarheid, energieverbruik, roaming, en beveiligingsrisico’s. Dit moet inzicht geven in de praktische inzetbaarheid van mobiele netwerken voor grootschalige gebouwtoepassingen.\newline\newline\newline\newline\newline\newline\newline

\end{multicols}
\end{document}